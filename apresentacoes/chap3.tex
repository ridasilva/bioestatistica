\documentclass[11pt]{beamer}
\usetheme{CambridgeUS}
\usepackage[utf8]{inputenc}
\usepackage{amsmath}
\usepackage{amsfonts}
\usepackage{amssymb}
\usepackage[
backend=biber,
style=alphabetic,
citestyle=authoryear
]{biblatex}

\newcommand\Fontvi{\fontsize{8}{7.2}\selectfont}

% Footnote without number
\newcommand\blfootnote[1]{%
  \begingroup
  \renewcommand\thefootnote{}\footnote{#1}%
  \addtocounter{footnote}{-1}%
  \endgroup
}
\newcommand{\comment}[1]{}

\addbibresource{stats.bib}
\title[Bioestatística II] %optional
{Variáveis Aleatórias Discretas}

\subtitle{CGF2022 - Bioestatística I}

\author[da Silva, Ricardo] % (optional, for multiple authors)
{R. ~R. ~da Silva\inst{1}}

\institute[FCFRP] % (optional)
{
  \inst{1}%
  Departamento de Ciências BioMoleculares\\
  Faculdade de Ciências Farmacêuticas

}

\date{\today} % (optional)

\titlegraphic{\includegraphics[width=5.8cm]{figs/logo_final}} 
\begin{document}

\begin{frame}
\titlepage
\end{frame}

\begin{frame}
\label{contents}
\frametitle{Sumário}
\tableofcontents
\end{frame}

\section{Introdução}
\setbeamercovered{transparent}
\begin{frame}
\frametitle{Resumo}
\begin{itemize}
\uncover<1->{
  \item Fenômeno aleatório: situação ou acontecimento cujos resultados não
  podem ser previstos com certeza.
}
\uncover<2->{
  \item Espaço amostral: conjunto de todos os resultados possíveis de um
  fenômeno aleatório, denotado por \(\Omega\).
}
\uncover<3->{
  \item Eventos: subconjuntos de \(\Omega\), denotado por \(A, B, \ldots\).
}
\uncover<4->{
  \item Conjunto vazio: conjunto sem eventos, denotado por \(\emptyset\).
}
\uncover<5->{
  \item União \(A \cup B\): ocorrência de pelo menos um dos eventos A ou B.
}
\uncover<6->{
  \item Intersecção \(A \cap B\): ocorrência simultânea de A e B.
}
\uncover<7->{
  \item Eventos disjuntos ou mutuamente exclusivos: \(A \cap B = \emptyset\).
}
\uncover<8->{
  \item Eventos complementares: \(A \cup A^{c} = \Omega\) e
  \(A \cap A^{c} = \emptyset\).
}
\end{itemize}
\end{frame}

\setbeamercovered{transparent}
\begin{frame}
\frametitle{Definição e Notação}

\begin{block}{Definição}
\begin{itemize}
\item
  \textbf{Variável aleatória} - Descrição numérica do resultado de um
  fenômeno aleatório.
\end{itemize}
\end{block}

\begin{block}{Notação}
\begin{itemize}
\item
  \(X\) denota a variável aleatória.
\item
  \(x\) denota os valores realizados da v.a.
\item
  Probabilidade de \(X\) assumir o valor \(x\) é denotada \(P(X = x)\).
\end{itemize}
\end{block}

\begin{block}{Exemplo}

\begin{itemize}
\item
  X: Número de alunos em uma sala de aula.
\item
  Uma possível realização \(x = 50\).
\end{itemize}
\end{block}
\end{frame}

\setbeamercovered{transparent}
\begin{frame}
\frametitle{Tipos de dados, espaço amostral e v.a}

\begin{itemize}
\item
  Tipos de dados

  \begin{enumerate}
  \def\labelenumi{\arabic{enumi}.}
  \item
    Dados na reta real, \(\Omega = \Re\).
  \item
    Dados estritamente positivos, \(\Omega = \Re_{+}\).
  \item
    Dados positivos com zeros, \(\Omega = \Re_{0} = [0,\infty)\).
  \item
    Proporções, \(\Omega = (0,1)\).
  \item
    Direções, \(\Omega = [0, 2 \pi)\).
  \item
    Contagens, \(\Omega = \mathbb{N}_{0} = \{0,1,2,\ldots \}\).
  \item
    Binomial, \(\Omega = = \{0,1,2,\ldots,m \}\).
  \end{enumerate}
\item
  Tipos de espaço amostral

  \begin{enumerate}
  \def\labelenumi{\arabic{enumi}.}
  \item
    \textbf{Espaço amostral Discreto}: Contêm apenas um número finito ou
    contável de elementos.
  \item
    \textbf{Espaço amostral Contínuo}: Contêm um número infinito de
    elementos.
  \end{enumerate}
\item
  Tipos de variáveis aleatórias

  \begin{enumerate}
  \def\labelenumi{\arabic{enumi}.}
  \item
    Variável aleatória é \textbf{contínua} se seu espaço amostral é
    contínuo.
  \item
    Variável aleatória é \textbf{dicreta} se seu espaço amostral é
    discreto.
  \end{enumerate}
\end{itemize}
\end{frame}

\setbeamercovered{transparent}
\begin{frame}
\frametitle{Descrição probabilística de v.a}

\begin{itemize}
\item
  Dada a realização de um experimento aleatório qualquer, com um certo
  espaço de probabilidade, desejamos estudar a \textbf{estrutura
  probabilística} de quantidades associadas à esse experimento.
\item
  Note que antes da realização de um experimento, \textbf{não sabemos
  seu resultado}, entretanto seu espaço amostral pode ser estabelecido
  com precisão.
\item
  Podemos atribuir probabilidades aos \emph{eventos} deste espaço
  amostral, dando origem a idéia de \textbf{distribuição de
  probabilidade}.
\item
  Em geral vamos distinguir a distribuição de probabilidade de v.a's
  discretas e contínuas.
\end{itemize}
\end{frame}

\section{Função de Probabilidade}
\setbeamercovered{transparent}
\begin{frame}
	
\end{frame}

\setbeamercovered{transparent}
\begin{frame}
\frametitle{Exemplo 3.3}

\begin{block}{Experimento}

Lançamento de duas moedas. \(X\) = número de resultados cara (C).


\begin{center}\includegraphics[width=0.6\linewidth]{figs/exp_moedas-crop} \end{center}
\end{block}
\end{frame}

\setbeamercovered{transparent}
\begin{frame}
\frametitle{Exemplo 3.3}

Podemos montar uma tabela de distribuição de frequência para a variável
aleatória \(X\) = número de resultados cara (C).

\begin{table}[h]
  \centering
    \begin{tabular}{ccc}
      \hline
      $X$ & Frequência ($f_i$) & Frequência relativa ($fr_i$) \\
      \hline
      0 & 1 & 1/4 \\
      1 & 2 & 2/4\\
      2 & 1 & 1/4\\
      \hline
      Total & 4 & 1 \\
      \hline
    \end{tabular}
\end{table}

Assim podemos associar a cada valor de \(X\) sua \textbf{probabilidade}
correspondente, como resultado das \textbf{frequências relativas}.
\begin{align*}
P[X=0] &= 1/4 \\
P[X=1] &= 2/4 = 1/2 \\
P[X=2] &= 1/4
\end{align*}
\end{frame}

\setbeamercovered{transparent}
\begin{frame}
\frametitle{Exemplo 3.3}

Dessa forma, a \textbf{distribuição de probabilidade} da variável
aleatória \(X\) = número de resultados cara (C) é a tabela:

\begin{table}[h]
    \centering
    \begin{tabular}{ccc}
      \hline
      $X$ & $P[X = x_i] = p(x_i)$ \\
      \hline
      0 & 1/4 \\
      1 & 1/2\\
      2 & 1/4\\
      \hline
      Total & 1 \\
      \hline
    \end{tabular}
\end{table}

Repare que as propriedades da função de probabilidade estão satisfeitas:

\begin{enumerate}
\def\labelenumi{\roman{enumi}.}
\item
  As probabilidades \(p(x_i)\) estão entre 0 e 1.
\item
  A soma de todas as probabilidades \(p(x_i)\) é 1.
\end{enumerate}
\end{frame}

\setbeamercovered{transparent}
\begin{frame}
\frametitle{Exemplo 3.1}

Com dados do último censo, a assistente social de um Centro de Saúde
constatou que para as famílias da região, \(20\%\) não tem filhos,
\(30\%\) tem um filho, \(35\%\) tem dois, e as restantes se dividem
igualmente entre três, quatro ou cinco filhos. Descreva a função de
probabilidade da v.a \(N\) definida como número de filhos.

\begin{table}[]
\begin{tabular}{lllllll} \hline
X      & 0    & 1    & 2    & 3 & 4 & 5 \\ \hline
P(X=x) & 0.20 & 0.30 & 0.35 & ? & ? & ? \\ \hline
\end{tabular}
\end{table}

Porém sabemos que
\[ \sum_{i=1}^6 P(X = x_i) = 0.20 + 0.30 + 0.35 + x + x + x = 1.\]

\begin{table}[]
\begin{tabular}{lllllll} \hline
X      & 0    & 1    & 2    & 3 & 4 & 5 \\ \hline
P(X=x) & 0.20 & 0.30 & 0.35 & 0.05 & 0.05 & 0.05 \\ \hline
\end{tabular}
\end{table}
\end{frame}

\setbeamercovered{transparent}
\begin{frame}
\frametitle{Exemplo 3.2}

Na construção de um certo prédio, as fundações devem atingir \(15\)
metros de profundidade e, para cada \(5\) metros de estacas colocadas, o
operador anota se houve alteração no ritmo de perfuração previamente
estabelecido. Essa alteração é resultado de mudanças para mais ou para
menos, na resistência do subsolo.

Nos dois casos, medidas corretivas serão necessárias, encarecendo o
custo da obra. Com base em avaliações geológicas, admite-se que a
probabilidade de ocorrência de alterações é de \(0.1\) para cada \(5\)
metros.

O custo básico inicial é de \(100\) UPC (unidade padrão de construção) e
será acrescida de \(50k\), com \(k\) representando o número de
alterações observadas. Como se comporta a v.a custo das obras de
fundações?
\end{frame}

%\setbeamercovered{transparent}
%\begin{frame}
%\frametitle{Exemplo 3.2}

%\begin{center}\includegraphics[width=1.15\linewidth]{figs/Arvore} \end{center}
%\end{frame}

\setbeamercovered{transparent}
\begin{frame}
\frametitle{Exemplo 3.2}


\end{frame}



\setbeamercovered{transparent}
\begin{frame}
\frametitle{Exemplo 3.2}

\begin{itemize}
\item
  Construindo a função de probabilidade.
\item
  V.a custo da obra.
\end{itemize}

\begin{table}[]
\begin{tabular}{lllllll} \hline
C      & 100     & 150    & 200     & 250 & \\ \hline
P(C=c) & 0.729 & 0.243  & 0.027 & 0.001  \\ \hline
\end{tabular}
\end{table}
\end{frame}

\setbeamercovered{transparent}
\begin{frame}
\frametitle{Função de distribuição de probabilidade}

\begin{itemize}
\item
  Em muitas situações, é útil calcularmos a probabilidade
  \textbf{acumulada} até um certo valor.
\item
  Definimos a \textbf{função de distribuição} ou \textbf{função
  acumulada de probablidade} de uma v.a \(X\) pela expressão:
\end{itemize}

\[
F(x) = P(X \leq x)
\]

para qualquer número real \(x\).
\end{frame}

\setbeamercovered{transparent}
\begin{frame}
\frametitle{Exemplo 3.5}

Uma população de 1000 crianças foi analisada num estudo para determinar
a efetividade de uma vacina contra um tipo de alergia.

No estudo, as crianças recebiam uma dose da vacina e, após um mês,
passavam por um novo teste. Caso ainda tivessem tido alguma reação
alérgica, recebiam outra dose da vacina. Ao fim de 5 doses todas as
crianças foram consideradas imunizadas.

Os resultados estão na tabela a seguir.

\begin{table}[ht]
\centering
\begin{tabular}{rrrrrr}
  \hline
 & 1 & 2 & 3 & 4 & 5 \\ 
  \hline
Freq. & 245 & 288 & 256 & 145 & 66 \\ 
   \hline
\end{tabular}
\end{table}

Para uma criança sorteado ao acaso qual a probabilidade dela ter
recebido \(2\) doses? E até \(2\) doses?
\end{frame}

\setbeamercovered{transparent}
\begin{frame}
\frametitle{Exemplo 3.5}

\begin{columns}
\begin{column}{0.5\textwidth}
Tabela de frequência:

\begin{table}[ht]
\centering
\begin{tabular}{rrrr}
  \hline
 & $n_i$ & $f_i$ & $f_{ac}$ \\ 
  \hline
1 & 245 & 0.245 & 0.245 \\ 
  2 & 288 & 0.288 & 0.533 \\ 
  3 & 256 & 0.256 & 0.789 \\ 
  4 & 145 & 0.145 & 0.934 \\ 
  5 & 66 & 0.066 & 1.000 \\ 
   \hline
Sum & 1000 & 1.000 &  \\ 
   \hline
\end{tabular}
\end{table}

\end{column}
\begin{column}{0.5\textwidth}  %%<--- here

Grafico de \(F(X)\):

\begin{center}\includegraphics[width=0.99\linewidth]{figs/unnamed-chunk-16-1} \end{center}

\end{column}
\end{columns}
\end{frame}

\setbeamercovered{transparent}
\begin{frame}
\frametitle{Exemplo 3.5}

Assim, \[
F(2) = P(X \leq 2) = P(X = 1) + P(X = 2) = 0,533.
\] Note que podemos escrever \[
F(x) = P(X \leq x) = 0,533 \quad \text{para} \quad 2 \leq x < 3.
\] Função de distribuição \small \[F(x) = \left\{
  \begin{array}{lr}
    0 & \text{se } x < 1 \\
    0,245 & \text{se } 1 \leq x < 2 \\
    0,533 & \text{se } 2 \leq x < 3 \\
    0,789 & \text{se } 3 \leq x < 4 \\
    0,934 & \text{se } 4 \leq x < 5 \\
    1 & \text{se } x \geq 5
  \end{array}
\right.
\]
\end{frame}

\setbeamercovered{transparent}
\begin{frame}
\frametitle{Exemplo 3.6}

Num estudo sobre a incidência de câncer foi registrado, para cada
paciente com esse diagnóstico, o número de casos de câncer em parentes
próximos (pais, irmãos, tios, filhos, primos e sobrinhos). A
distribuição de frequência para 26 pacientes é a seguinte:

\begin{table}[ht]
\centering
\begin{tabular}{rrrrrrr}
  \hline
 & 0 & 1 & 2 & 3 & 4 & 5 \\ 
  \hline
$n_i$ & 4 & 4 & 6 & 6 & 2 & 4 \\ 
   \hline
\end{tabular}
\end{table}

Estudos anteriores assumem que a incidência de câncer em parentes
próximos pode ser teoricamente modelada pela seguinte função discreta de
probabilidade:

\begin{table}[ht]
\centering
\begin{tabular}{rrrrrrr}
  \hline
 & 0 & 1 & 2 & 3 & 4 & 5 \\ 
  \hline
$p_i$ & 0.1 & 0.1 & 0.3 & 0.3 & 0.1 & 0.1 \\ 
   \hline
\end{tabular}
\end{table}

Os dados observados concordam com o modelo teórico?
\end{frame}


\setbeamercovered{transparent}
\begin{frame}
\frametitle{Exemplo 3.6}

O número de observações de incidência esperado seguindo o modelo teórico
é calculado como \[e_i = n \times p_i.\]

\begin{columns}
\begin{column}{0.5\textwidth}
\begin{table}[ht]
\centering
\begin{tabular}{rrr}
  \hline
 & $n_i$ & $e_i$ \\ 
  \hline
0 & 4 & 2.6 \\ 
  1 & 4 & 2.6 \\ 
  2 & 6 & 7.8 \\ 
  3 & 6 & 7.8 \\ 
  4 & 2 & 2.6 \\ 
  5 & 4 & 2.6 \\ 
   \hline
Sum & 26 & 26.0 \\ 
   \hline
\end{tabular}
\end{table}

\end{column}
\begin{column}{0.5\textwidth}  %%<--- here

\begin{center}\includegraphics[width=0.99\linewidth]{figs/esperado-1} \end{center}

\end{column}
\end{columns}
\end{frame}

\section{Distribuições discretas de probabilidade}
\setbeamercovered{transparent}
\begin{frame}
\frametitle{Modelo Uniforme Discreto}

\textbf{Definição:} Seja \(X\) uma v.a assumindo valores
\(1, 2, \ldots, k\). Dizemos que \(X\) segue o modelo \textbf{Uniforme
Discreto} se atribui a mesma probabilidade \(1/k\) a cada um desses
\(k\) valores.

Então, sua função de probabilidade é dada por \[
P[X = j] =  \frac{1}{k}, \quad j = 1, 2, \ldots, k.
\]

\textbf{Notação}: \(X \sim \text{U}_D[1,k]\)
\end{frame}

\setbeamercovered{transparent}
\begin{frame}
\frametitle{Modelo Uniforme Discreto}

\begin{center}\includegraphics[width=0.8\linewidth]{figs/unnamed-chunk-20-1} \end{center}
\end{frame}

\setbeamercovered{transparent}
\begin{frame}
\frametitle{Exemplo 3.7}

Uma rifa tem 100 bilhetes numerados de 1 a 100. Tenho 5 bilhetes
consecutivos numerados de 21 a 25 e meu colega tem outros 5 bilhetes,
com os números 1, 11, 29, 68 e 93. Quem tem maior possibilidade de ser
sorteado?
\end{frame}

\setbeamercovered{transparent}
\begin{frame}
\frametitle{Modelo Bernoulli}

\textbf{Definição:} Uma variável aleatória \(X\) segue o modelo
Bernoulli se assume apenas os valores 0 (``fracasso'') ou 1
(``sucesso''). Sua função de probabilidade é dada por \[
P[X = x] = p^x (1-p)^{1-x}, \quad \quad x = 0, 1
\] onde o parâmetro \(0 \leq p \leq 1\) é a probabilidade de sucesso.

\textbf{Notação:} \(X \sim \text{Ber}(p)\)
\end{frame}

\setbeamercovered{transparent}
\begin{frame}
\frametitle{Modelo Bernoulli}

\begin{center}\includegraphics[width=0.99\linewidth]{figs/unnamed-chunk-21-1} \end{center}
\end{frame}

\setbeamercovered{transparent}
\begin{frame}
\frametitle{Exemplo}

No lançamento de uma moeda, considere cara como o evento de sucesso.
Qual a probabilidade de sair cara, sendo que \(p = 1/2\)?

\[ X = \left\{
  \begin{array}{ll}
    1, & \quad \text{cara}\\
    0, & \quad \text{coroa}
\end{array} \right.\]

Temos que

\begin{table}[h]
  \centering
  \begin{tabular}{ccc}
    \hline
    $X$ & $P[X=x]$ & $p = 1/2$ \\
    \hline
    0 & $1-p$ & $1/2$ \\
    1 & $p$ & $1/2$ \\
    \hline
  \end{tabular}
\end{table}
\end{frame}

\setbeamercovered{transparent}
\begin{frame}
\frametitle{Modelo Binomial}

\textbf{Definição:} Seja um experimento realizado dentro das seguintes
condições:

\begin{enumerate}
\def\labelenumi{\roman{enumi}.}
\item
  São realizados \(n\) ``ensaios'' de Bernoulli independentes.
\item
  Cada ensaio só pode ter dois resultados possíveis: ``sucesso'' ou
  ``fracasso''.
\item
  A probabilidade \(p\) de sucesso em cada ensaio é constante.
\end{enumerate}

Vamos associar a v.a \(X\) o número de sucessos em \(n\) ensaios de
Bernoulli. Portanto \(X\) poderá assumir os valores \(0, 1, \ldots, n\).
\end{frame}

\setbeamercovered{transparent}
\begin{frame}
\frametitle{Modelo Binomial}

Vamos determinar a distribuição de probabilidade de \(X\), através da
probabilidade de um número genérico \(x\) de sucessos.

Suponha que ocorram sucessos (1) apenas nas \(k\) primeiras provas, e
fracassos (0) nas \(n-k\) provas restantes \[
\underbrace{1,1,1,..., 1}_{k}, \underbrace{0,0,0, ..., 0}_{n-k}.
\] Como as provas são independentes, a probabilidade de ocorrência de
\(k\) sucessos em \(n\) tentativas é uma extensão do modelo de Bernoulli
para \(n\) ensaios, ou seja, \[
\underbrace{p\cdot p\cdot p \cdots p}_{k} \cdot
\underbrace{(1-p)\cdot (1-p)\cdot (1-p) \cdots (1-p)}_{n-k} =
 p^k (1-p)^{n-k}.
\]
\end{frame}

\setbeamercovered{transparent}
\begin{frame}
\frametitle{Modelo Binomial}

Porém, o evento: ``\(k\) sucessos em \(n\) provas'' pode ocorrer de
diferentes maneiras (ordens) distintas, todas com a mesma probabilidade.

Como o número de ordens é o número de combinações de \(n\) elementos
tomados \(k\) a \(k\), então a probabilidade de ocorrerem \(k\) sucessos
em \(n\) provas de Bernoulli será a distribuição binomial, dada
por \[
P[X = x] = \binom{n}{k} p^k (1-p)^{n-k}, \quad \quad x = 0, 1, \ldots, n
\] onde \[
\binom{n}{k} = \frac{n!}{k!(n-k)!}
\] é o \textbf{coeficiente binomial}, que dá o número total de
combinações possíveis de \(n\) elementos, com \(x\) sucessos.

\textbf{Notação:} \(X \sim B(n,p)\)
\end{frame}

\setbeamercovered{transparent}
\begin{frame}
\frametitle{Modelo Binomial}

\begin{center}\includegraphics[width=0.9\linewidth]{figs/unnamed-chunk-22-1} \end{center}
\end{frame}

\setbeamercovered{transparent}
\begin{frame}
\frametitle{Modelo Binomial}

\begin{itemize}
\item
  As probabilidades são completamente caracterizadas pela informação dos
  \textbf{parâmetros}.
\item
  Por exemplo, para calcular a probabilidade de 3 sucessos de uma
  \(B(12, 0.4)\) fazemos \[
  P[X = 3] = \binom{12}{3} 0.4^3 0.6^{9} =
   \frac{12!}{3!9!} 0.4^3 0.6^{9} = 0.142
  \] Diversos programas computacionais fazem esse cálculo facilmente.
  Por exemplo, com o Python.
\end{itemize}
\end{frame}

\setbeamercovered{transparent}
\begin{frame}
\frametitle{Exemplo 3.9}

O escore de um teste internacional de proficiência na língua inglesa
varia de \(0\) a \(700\) pontos, com mais pontos indicando um melhor
desempenho. Informações, coletadas durante vários anos, permite
estabelecer o seguinte modelo para o desempenho no teste:

\begin{table}[]
\centering
\scriptsize
\begin{tabular}{|l|l|l|l|l|l|l|}
\hline
Pontos & $[0,200)$ & $[200,300)$ & $[300,400)$ & $[400,500)$ & $[500,600)$ & $[600,700)$ \\ \hline
$p_i$   & $0.06$   & $0.15$       & $0.16$       & $0.25$     & $0.28$   & $0.10$        \\ \hline
\end{tabular}
\end{table}

Várias universidades americanas, exigem um escore mínimo de \(600\)
pontos para aceitar candidatos de países de língua não inglesa. De um
grande grupo de estudantes brasileiros que prestaram o último exame,
escolhemos ao acaso \(20\) deles. Qual é a probabilidade de no máximo
\(3\) atenderem ao requisito mínimo?
\end{frame}

\setbeamercovered{transparent}
\begin{frame}
\frametitle{Exemplo 3.9}


\end{frame}


\setbeamercovered{transparent}
\begin{frame}
\frametitle{Modelo Geométrico}

\textbf{Definição:} Considere o número (\(k\)) de ensaios Bernoulli
\textbf{que precedem o primeiro sucesso}. Nesse caso, dizemos que a v.a
\(X\) tem distribuição Geométrica de parâmetro \(p\), e sua função de
probabilidade tem a forma \[
P(X = k) = p(1-p)^{k}, \quad k = 0, 1, 2, \ldots .
\] onde \(0 \leq p \leq 1\) é a probabilidade de sucesso.

\textbf{Notação:} \(X \sim \text{G}(p)\).
\end{frame}

\setbeamercovered{transparent}
\begin{frame}
\frametitle{Modelo Geométrico}

\begin{center}\includegraphics[width=0.9\linewidth]{figs/unnamed-chunk-27-1} \end{center}
\end{frame}

\setbeamercovered{transparent}
\begin{frame}
\frametitle{Exemplo 3.11}

Uma linha de produção está sendo analisada para efeito de controle da
qualidade das peças produzidas. Tendo em vista o alto padrão requerido,
a produção é interrompida para regulagem toda vez que uma peça
defeituosa é observada. Se \(0.01\) é a probabilidade da peça ser
defeituosa, estude o comportamento da variável \(Q\), quantidade de
peças boas produzidas antes da primeira defeituosa.
\end{frame}

\setbeamercovered{transparent}
\begin{frame}
\frametitle{Exemplo 3.11}

\(P(Q = k) = 0.01 \times 0.99^{k}, \quad k = 0, 1, 2, \ldots\)

\begin{center}\includegraphics[width=0.9\linewidth]{figs/unnamed-chunk-28-1} \end{center}
\end{frame}


\setbeamercovered{transparent}
\begin{frame}
\frametitle{Modelo Poisson}

\textbf{Definição:} Seja um experimento realizado nas seguintes
condições:

\begin{enumerate}
\def\labelenumi{\roman{enumi}.}
\item
  As ocorrências são independentes.
\item
  As ocorrências são aleatórias.
\item
  A variável aleatória \(X\) é o número de ocorrências de um evento
  \textbf{ao longo de algum intervalo} (de tempo ou espaço).
\end{enumerate}

Denominamos esse experimento de \textbf{processo de Poisson}.

Vamos associar a v.a \(X\) o número de ocorrências em um intervalo.
Portanto \(X\) poderá assumir os valores \(0, 1, \ldots\) (sem limite
superior).
\end{frame}

\setbeamercovered{transparent}
\begin{frame}
\frametitle{Modelo Poisson}

Uma v.a \(X\) segue o modelo de Poisson se surge a partir de um processo
de Poisson, e sua \textbf{função de probabilidade} for dada por \[
P(X = k) = \frac{e^{-\lambda}\lambda^k}{k!}, \quad k = 0, 1, 2, \ldots .
\] onde o parâmetro \(\lambda > 0\) é a taxa média de ocorrências em um
intervalo de tempo ou espaço.

Notação: \(X \sim P(\lambda)\).
\end{frame}

\setbeamercovered{transparent}
\begin{frame}
\frametitle{Modelo Poisson}

\begin{center}\includegraphics[width=0.9\linewidth]{figs/unnamed-chunk-29-1} \end{center}
\end{frame}

\setbeamercovered{transparent}
\begin{frame}
\frametitle{Exemplo 3.12}

A emissão de partículas radioativas têm sido modelada através de uma
distribuição de Poisson, com o valor do parâmetro dependendo da fonte
utilizada.

Suponha que o número de partículas alfa, emitidas por minuto, seja uma
variável aleatória seguindo o modelo Poisson com parâmetro 5, isto é, a
taxa média de ocorrência é de 5 emissões a cada minuto.

Calcule a probabilidade de haver mais de \(2\) emissões em um minuto.
\end{frame}

\setbeamercovered{transparent}
\begin{frame}
\frametitle{Exemplo 3.12}

Note que \(P(A > 2) = 1 - P(A \leq 2)\), portanto \begin{align*}
P(A > 2) &= \sum_{a = 3}^{\infty} P(A = a) = 1 - P(A \leq 2) \\
&= 1 - \sum_{a =0}^2 \frac{e^{-5}5^a}{a!} = 0.875
\end{align*}
\end{frame}


\setbeamercovered{transparent}
\begin{frame}
\frametitle{Modelo hipergeométrico}

\textbf{Definição:} Considere um conjunto de \(n\) objetos dos quais
\(m\) são do tipo \(I\) e \(n-m\) são do tipo \(II\). Para um sorteio de
\(r\) objetos \(r < n\), feito ao acaso e \textbf{sem reposição}, defina
\(X\) como o número de objetos de tipo \(I\) selecionados.

Diremos que a v.a \(X\) segue o modelo Hipergeométrico e sua função de
probabilidade é dada por \[
P[X = k] = \frac{\dbinom{m}{k} \dbinom{n-m}{r-k}}{\dbinom{n}{r}},
\] onde \(k = \max{\{0, r - (n-m)\}} \leq k \leq \min{\{r, m\}}\).

\textbf{Notação:} \(X \sim \text{HG}(m, n, r)\).
\end{frame}

\setbeamercovered{transparent}
\begin{frame}
\frametitle{Exemplo}

Considere que, num lote de 20 peças, existam 4 defeituosas.
Selecionando-se 5 dessas peças, sem reposição, qual seria a
probabilidade de 2 defeituosas terem sido escolhidas?
\end{frame}

\setbeamercovered{transparent}
\begin{frame}
\frametitle{Exemplo}

Pelo enunciado, sabemos que \(n = 20\), \(m = 4\), \(r = 5\). Podemos
calcular a probabilidade de \(X = 2\) como

\begin{align*}
P[X = 2] &= \frac{\dbinom{m}{k} \dbinom{n-m}{r-k}}{\dbinom{n}{r}}
  = \frac{\dbinom{4}{2} \dbinom{20-4}{5-2}}{\dbinom{20}{5}} = 0.217
\end{align*}
\end{frame}



\setbeamercovered{transparent}
\begin{frame}
\frametitle{Exercícios recomendados}

\begin{itemize}
\item
  Seção \(3.1\) - \(3,5\).
\item
  Seção \(3.2\) - \(2,3,5\) e \(6\).
\item
  Seção \(3.3\) - \(1,2,3,4,5\) e \(6\).
\end{itemize}
\end{frame}

\end{document}