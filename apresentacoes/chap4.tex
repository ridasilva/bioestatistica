\documentclass[11pt]{beamer}
\usetheme{CambridgeUS}
\usepackage[brazil]{babel}
\usepackage[utf8]{inputenc}
\usepackage[T1]{fontenc}
\usepackage{textcomp}
\usepackage{amsmath}
\usepackage{amsfonts}
\usepackage{amssymb}
\usepackage[
backend=biber,
style=alphabetic,
citestyle=authoryear
]{biblatex}

% Footnote without number
\newcommand\blfootnote[1]{%
  \begingroup
  \renewcommand\thefootnote{}\footnote{#1}%
  \addtocounter{footnote}{-1}%
  \endgroup
}

% Reduced font size
\newcommand\Fontvi{\fontsize{8}{7.2}\selectfont}

\addbibresource{stats.bib}
\title[Bioestatística II] %optional
{Medidas Resumo}

\subtitle{CGF2046 - Bioestatística II}

\author[da Silva, Ricardo] % (optional, for multiple authors)
{R. ~R. ~da Silva\inst{1}}

\institute[FCFRP] % (optional)
{
  \inst{1}%
  Departamento de Ciências BioMoleculares\\
  Faculdade de Ciências Farmacêuticas

}

\date{\today} % (optional)

\titlegraphic{\includegraphics[width=5.8cm]{figs/logo_final}} 


\begin{document}

\begin{frame}
\titlepage
\end{frame}

\begin{frame}
\label{contents}
\frametitle{Sumário}
\tableofcontents
\end{frame}

\section{Introdução}
\setbeamercovered{transparent}
\begin{frame}
\frametitle{Introdução}

Características importantes de qualquer conjunto de dados ou de uma
variável aleatória

\begin{itemize}
\item
  Centro
\item
  Variação
\item
  Distribuição
\item
  Valores atípicos
\end{itemize}
\end{frame}


\setbeamercovered{transparent}
\begin{frame}
\frametitle{Introdução}

Características importantes de qualquer conjunto de dados ou de uma
variável aleatória

\begin{itemize}
\item
  Centro
\item
  Variação
\item
  Distribuição
\item
  Valores atípicos
\end{itemize}

Classificaremos as medidas descritivas em dois grupos

\begin{itemize}
\item
  de posição
\item
  de dispersão
\end{itemize}
\end{frame}

\section{Medidas de posição}
\setbeamercovered{transparent}
\begin{frame}
\frametitle{Definição}

\begin{itemize}
\item
  medidas de posição central

  \begin{itemize}
  \item
    úteis para \textbf{resumo} e \textbf{análise} de dados

    \begin{itemize}
    \item
      Média, Mediana, Moda
    \end{itemize}
  \end{itemize}
\item
  outras medidas de posição

  \begin{itemize}
  \item
    extremos: mínimo, máximo
  \item
    quantis: 1\(^\circ\) quartil, 3\(^\circ\) quartil, percentil 5\%,
    entre outras
  \end{itemize}
\end{itemize}
\end{frame}

\setbeamercovered{transparent}
\begin{frame}
\frametitle{Moda}

Valor \textbf{mais frequente} em um conjunto de dados

\begin{itemize}
\item
  Dependendo do conjunto de dados, ele pode ser

  \begin{itemize}
  \item
    \textbf{Sem moda} quando nenhum valor se repete
  \item
    \textbf{Unimodal} quando existe apenas um valor repetido com maior
    frequência
  \item
    \textbf{Bimodal} quando existem dois valores com a mesma maior
    frequência
  \item
    \textbf{Multimodal} quando mais de dois valores se repetem com a
    mesma frequência
  \end{itemize}
\end{itemize}
\end{frame}

\setbeamercovered{transparent}
\begin{frame}
\frametitle{Moda}

Valor \textbf{mais frequente} em um conjunto de dados

\begin{itemize}
\item
  Dependendo do conjunto de dados, ele pode ser

  \begin{itemize}
  \item
    \textbf{Sem moda} quando nenhum valor se repete
  \item
    \textbf{Unimodal} quando existe apenas um valor repetido com maior
    frequência
  \item
    \textbf{Bimodal} quando existem dois valores com a mesma maior
    frequência
  \item
    \textbf{Multimodal} quando mais de dois valores se repetem com a
    mesma frequência
  \end{itemize}
\end{itemize}

Valor com \textbf{maior probabilidade} de ocorrer numa \textbf{VA
discreta}

\begin{itemize}
\item
  Ex.: lançamento de duas moedas

  \begin{itemize}
  \item
    \(X\): número de caras, \(X = \{0, 1, 2\}\)
  \item
    \(P(x)\) = \(0.25\), \(0.5\) e \(0.25\), respectivamente
  \item
    moda: 1
  \end{itemize}
\end{itemize}

\end{frame}

\setbeamercovered{transparent}
\begin{frame}
\frametitle{Mediana}

O \textbf{valor do meio} da amostra \textbf{ordenada}

\begin{itemize}
\item
  Separa o conjunto de dados em duas partes iguais, 50\% abaixo e 50\%
  acima
\end{itemize}

Observações ordenadas:

\begin{itemize}
\item
  a menor observação por \(x_{(1)}\), a segunda por \(x_{(2)}\), e assim
  por diante: \[
  x_{(1)} \leq x_{(2)} \leq \, \cdots \, \leq x_{(n-1)} \leq x_{(n)}
  \]
\end{itemize}

As observações odenadas são chamadas de \textbf{estatísticas de ordem}

\begin{itemize}
\item
  \(x_{(1)}\) é o mínimo da amostra
\item
  \(x_{(n)}\) é o máximo da amostra
\end{itemize}
\end{frame}

\setbeamercovered{transparent}
\begin{frame}
\frametitle{Média de dados brutos}

Divide-se a soma de todos os dados pelo número total deles:

\[
\bar{x}_{obs} = \frac{x_1 + x_2 + \cdots + x_n}{n} = \frac{\sum_{i=1}^n
x_i}{n}.
\]
\end{frame}

\setbeamercovered{transparent}
\begin{frame}
\frametitle{Média de dados agrupados}

Soma dos produtos dos valores pelas respectivas frequências e divide
pela frequência total

\[
\bar{x}_{obs} = \frac{n_1 x_1 + n_2 x_2 + \cdots + n_k x_k}{n_1 + n_2 +
\cdots + n_k} = \frac{\sum_{i=1}^k n_i x_i}{n}.
\]
\end{frame}

\setbeamercovered{transparent}
\begin{frame}
\frametitle{Exemplo: média de dados discretos agrupados}

Considere a tabela de frequência abaixo:

\begin{table}[h]
\centering
    \small
    \begin{tabular}{ccc}
      \hline
      \textbf{Número} & \textbf{$n_i$} & \textbf{$f_i$} \\
      \hline
      0 & 4 & 0,20 \\
      1 & 5 & 0,25 \\
      2 & 7 & 0,35 \\
      3 & 3 & 0,15 \\
      5 & 1 & 0,05 \\
      \hline
      \textbf{Total} & 20 & 1 \\
      \hline
    \end{tabular}
\end{table}

A média é calculada por: \begin{align*}
\bar{x}_{obs} &= \frac{ 0 \cdot 4 + 1 \cdot 5 + 2 \cdot 7 + 3 \cdot 3 + 5 \cdot 1 }{4 + 5 + 7 + 3 + 1}\\
  &= \frac{33}{20} \\
  &= 1,65
\end{align*}
\end{frame}

\setbeamercovered{transparent}
\begin{frame}
\frametitle{Exemplo: média de dados agrupados em classes}

Usar \textbf{ponto médio} de cada classe e respectivas frequências

\begin{table}[h]
    \centering
    \small
    \begin{tabular}{lccc}
      \hline
      \textbf{Classe} & \textbf{PM = $x_i$} & \textbf{$n_i$} &
    \textbf{$f_i$} \\
      \hline
      $[4,8)$ & 6 & 10 & 0,278 \\
      $[8,12)$ & 10 & 12 & 0,333 \\
      $[12,16)$ & 14 & 8 & 0,222 \\
      $[16,20)$ & 18 & 5 & 0,139 \\
      $[20,24)$ & 22 & 1 & 0,028 \\
      \hline
      \textbf{Total} & 36 & 1 \\
      \hline
    \end{tabular}
\end{table}

Considerando os \textbf{pontos médios} de cada classe, a média é
calculada por \begin{align*}
\bar{x}_{obs} &= \frac{ (6 \cdot 10 + 10 \cdot 12 + \cdots + 22 \cdot 1) }{10 + 12 + 8 + 5 + 1}\\
  &= \frac{404}{36} \\
  &= 11,22
\end{align*}
\end{frame}

\setbeamercovered{transparent}
\begin{frame}
\frametitle{Exemplo 4.1}

Suponha que parafusos a serem utilizados em tomadas elétricas são
embalados em caixas rotuladas como contendo \(100\) unidades. Em uma
construção, \(10\) caixas de um lote tiveram o número de parafusos
contados, fornecendo os valores:

\(98\), \(102\), \(100\), \(100\), \(99\), \(97\), \(96\), \(95\),
\(99\) e \(100\)

Calcular média, mediana e moda.

\begin{itemize}
\item
  \(\bar{x}_{obs} = 98.6.\)
\item
  \(md_{obs} = 99.\)
\item
  \(mo_{obs} = 100.\)
\end{itemize}
\end{frame}

\setbeamercovered{transparent}
\begin{frame}
\frametitle{Média e mediana}

Notar a influência de valores extremos na média (se ao invés de 95, o
valor fosse 45):

\begin{center}
\texttt{95 96 97 98 99 99 100 100 100 102} $\quad \Rightarrow \quad $
  $\bar{x}_{obs} = 98,6$ e $Md = 99$
\end{center}
\begin{center}
\texttt{45 96 97 98 99 99 100 100 100 102} $\quad \Rightarrow \quad $
  $\bar{x}_{obs} = 93,6$ e $Md = 99$
\end{center}

Devido a esse fato, a mediana é uma medida de posição central
\textbf{robusta}, ou seja, \emph{não influenciada por valores extremos}.
\end{frame}

\setbeamercovered{transparent}
\begin{frame}
\frametitle{Média, mediana e moda}

\begin{center}\includegraphics[width=0.9\linewidth]{figs/medidas-crop} \end{center}
\end{frame}

\setbeamercovered{transparent}
\begin{frame}
\frametitle{Exemplo 4.4}

Um estudante está procurando um estágio para o próximo ano. As
companhias A e B têm programas de estágios e oferecem uma remuneração
por \(20\) horas semanais com as seguintes características.

\begin{table}
\centering
\begin{tabular}{lcc}
\hline
Companhia   & A     & B  \\ \hline
média       & $2,5$ & $2,0$ \\
mediana     & $1,7$ & $1,9$ \\
moda        & $1,5$ &  $1,9$ \\ \hline
\end{tabular}
\end{table}

Qual companhia você escolheria?
\end{frame}

\setbeamercovered{transparent}
\begin{frame}
\frametitle{Exemplo 4.3}

Foram coletadas \(150\) observações da variável \(X\), representando o
número de vestibulares FUVEST (um por ano) que um mesmo estudande
prestou. Com os dados da tabela abaixo, calcule as medidas de posição de
\(X\).

\begin{table}
\centering
\begin{tabular}{ll}
\hline
\multicolumn{1}{l}{X} & \multicolumn{1}{l}{$n_i$} \\ \hline
1                       & 75                        \\
2                       & 47                        \\
3                       & 21                        \\
4                       & 7                        \\ \hline
\end{tabular}
\end{table}

Suponha ainda que o interesse é estudar o gasto dos alunos associado com
as despesas do vestibular. Para simplificar, suponha que se atribui para
cada aluno, uma despesa fixa de R\$ \(1300,00\) relativa a preparação e
mais R\$ \(50\) para cada vestibular prestado. Calcule as medidas de
posição central para a variável \(D\) (despesa com vestibular).
\end{frame}

\setbeamercovered{transparent}
\begin{frame}
\frametitle{Exemplo 4.3}
\begin{columns}[T]
\begin{column}{5cm}
\begin{table}
\begin{tabular}{ll}
\hline
\multicolumn{1}{l}{X} & \multicolumn{1}{l}{$n_i$} \\ \hline
1                       & 75                        \\
2                       & 47                        \\
3                       & 21                        \\
4                       & 7                        \\ \hline
\end{tabular}
\end{table}
\end{column}
\begin{column}{10cm}
\end{column}
\end{columns}
\end{frame}

\setbeamercovered{transparent}
\begin{frame}
\frametitle{Medidas de posição para VAs discretas}

Sabemos que a descrição completa do comportamento de uma VA discreta é
feita através de sua \textbf{função de probabilidade}.

Assim como fizemos para um conjunto de dados qualquer, podemos obter as
medidas de posição para qualquer variável aleatória.

Lembrando que se os possíveis valores de uma VA \(X\) são
\(x_1, x_2, \ldots, x_k\), com correspondentes probabilidades
\(p_1, p_2, \ldots, p_k\), então as medidas de posição podem ser
definidas a seguir.
\end{frame}

\setbeamercovered{transparent}
\begin{frame}
\frametitle{Medidas de posição para VAs discretas}

A Média é chamada de \textbf{valor esperado} ou \textbf{esperança} \[
E(X) = \sum_{i=1}^k x_i p_i.
\] A Mediana é o valor \(Md\) que satisfaz as seguintes condições \[
P(X \leq Md) \geq 1/2 \quad \text{e} \quad P(X \geq Md) \geq 1/2.
\] A Moda é o valor (ou valores) com maior probabilidade de ocorrência
\[
P(X = Mo) = \max\{p_1, p_2, ..., p_k\}.
\]
\end{frame}

\setbeamercovered{transparent}
\begin{frame}
\frametitle{Exemplo 4.5}

Considere a VA \(X\) com a seguinte função discreta de probabilidade:

\begin{table}
\centering
\begin{tabular}{l|llll}
$X$    & -5  & 10  & 15  & 20  \\ \hline
$p_i$ & 0.3 & 0.2 & 0.4 & 0.1 \\
\end{tabular}
\end{table}

Calcule as medidas de tendência central.
\end{frame}

\setbeamercovered{transparent}
\begin{frame}
\frametitle{Exemplo 4.6}

Considere uma VA \(X\) com função de probabilidade dada por

\begin{table}
\centering
\begin{tabular}{l|lllll}
X    & 2  & 5  & 8  & 15 & 20  \\ \hline
$p_i$ & 0.1 & 0.3 & 0.2 & 0.2 & 0.2
\end{tabular}
\end{table}

Calcule as medidas de posição para a VA \(Y = 5X - 10\).
\end{frame}

\section{Medidas de dispersão}
\setbeamercovered{transparent}
\begin{frame}
\frametitle{Introdução}

O resumo de um conjunto de dados exclusivamente por uma medida de
centro, \textbf{esconde} toda a informação sobre a variabilidade do
conjunto de observações.

Não é possível analisar um conjunto de dados apenas através de uma
medida de tendência central.

Por isso precisamos de medidas que resumam a \textbf{variabilidade} dos
dados em relação à um valor central.
\end{frame}

\setbeamercovered{transparent}
\begin{frame}
\frametitle{Exemplo: mesma média, diferente dispersão}

\begin{center}\includegraphics[width=0.8\linewidth]{figs/unnamed-chunk-4-2.pdf} \end{center}
\end{frame}

\setbeamercovered{transparent}
\begin{frame}
\frametitle{Exemplo}

Cinco grupos de alunos se submeteram a um teste, obtendo as seguintes
notas

\begin{table}[htbp]
    \centering
    \begin{tabular}{crr}
      \hline
      \textbf{Grupo} & \textbf{Notas} & $\bar{x}$ \\ \hline
      A & 3, 4, 5, 6, 7 & 5 \\
      B & 1, 3, 5, 7, 9 & 5\\
      C & 5, 5, 5, 5, 5 & 5\\
      D & 3, 5, 5, 7 & 5\\
      E & 3, 5, 5, 6, 6 & 5\\
      \hline
    \end{tabular}
\end{table}

O que a média diz a respeito das notas quando comparamos os grupos?
\end{frame}

\setbeamercovered{transparent}
\begin{frame}
\frametitle{Definição}

São medidas estatísticas que caracterizam o quanto um conjunto de dados
está disperso em torno de sua tendência central.

Ferramentas para \textbf{resumo} e \textbf{análise} de dados:

\begin{itemize}
\item
  Amplitude
\item
  Desvio-médio (ou mediano)
\item
  Variância
\item
  Desvio-padrão
\item
  Coeficiente de Variação
\end{itemize}
\end{frame}

\setbeamercovered{transparent}
\begin{frame}
\frametitle{Amplitude}

A \textbf{amplitude} de um conjunto de dados é a diferença entre o maior
e o menor valor: \[
\Delta = \max - \min = x_{(n)} - x_{(1)}
\]

\begin{table}[htbp]
    \centering
    \begin{tabular}{crr}
      \hline
      \textbf{Grupo} & \textbf{Notas} & $\Delta$ \\ \hline
      A & 3, 4, 5, 6, 7 & 4 \\
      B & 1, 3, 5, 7, 9 & 8\\
      C & 5, 5, 5, 5, 5 & 0\\
      D & 3, 5, 5, 7 & 4\\
      E & 3, 5, 5, 6, 6 & 3\\
      \hline
    \end{tabular}
\end{table}

\begin{itemize}
\item
  \textbf{Apenas} usar máximo e mínimo torna \textbf{sensível} a valores
  extremos

  \begin{itemize}
  \item
    Melhor medida de variabilidade: considerar \textbf{todos os dados
    disponíveis}
  \item
    \textbf{Desvio} de cada valor em relação à uma medida de posição
    central (média ou mediana)
  \end{itemize}
\end{itemize}
\end{frame}

\setbeamercovered{transparent}
\begin{frame}
\frametitle{Desvio médio e mediano}

Um \textbf{resumo} da variabilidade: \textbf{média} dos desvios
\textbf{absolutos}

\begin{itemize}
\item
  \textbf{Desvio mediano}: a \textbf{mediana} como medida de posição
  central \[
  \text{desvio mediano} = \frac{1}{n}\sum_{i=1}^n | x_i - md_{obs}|.
  \]
\item
  \textbf{Desvio médio}: a \textbf{média} como medida de posição central
  \[
  \text{desvio médio} = \frac{1}{n}\sum_{i=1}^n | x_i - \bar{x}_{obs}|.
  \]
\end{itemize}
\end{frame}

\setbeamercovered{transparent}
\begin{frame}
\frametitle{Exemplo: Desvio médio}

Considere as notas do grupo A do exemplo acima (\(\bar{x}_{obs} = 5\))

O desvio médio (DM) pode ser calculado da seguinte forma:

\begin{table}[htbp]
 \centering
 \small
 \begin{tabular}{ccc}
 \hline
 \textbf{Grupo A} & $x_i - \bar{x}$ & $|x_i - \bar{x}|$ \\ \hline
 3 & -2 & 2 \\
 4 & -1 & 1\\
 5 & 0 & 0\\
 6 & 1 & 1\\
 7 & 2 & 2\\
 \hline
 Soma & 0 & 6\\
 \hline
 \end{tabular}
\end{table}

\(\text{DM} = \frac{1}{n}\sum_{i=1}^n | x_i - \bar{x}_{obs}| = \frac{6}{5} = 1,2\)

O desvio médio é baseado em uma operação \textbf{não algébrica}
(módulo), o que torna mais difícil o estudo de suas propriedades.
\end{frame}

\setbeamercovered{transparent}
\begin{frame}
\frametitle{Variância e desvio-padrão de um conjunto de dados}

Uma alternativa melhor é usar a \textbf{soma dos quadrados dos desvios},
que dá origem à \textbf{variância} de um conjunto de dados \[
var_{obs} = \frac{1}{n}\sum_{i=1}^n (x_i - \bar{x}_{obs})^2
\] Para manter a mesma unidade de medida dos dados originais, definimos
o \textbf{desvio padrão} como \[
dp_{obs} = \sqrt{var_{obs}}
\] Uma expressão alternativa da variância (mais fácil de calcular) é \[
var_{obs} = \frac{1}{n}\sum_{i=1}^n x_i^2 - \bar{x}_{obs}^2
\]
\end{frame}

\setbeamercovered{transparent}
\begin{frame}
\frametitle{Exemplo}

No exemplo anterior

\begin{table}[htbp]
 \centering
 \small
\begin{tabular}{ccccc}
 \hline
 \textbf{Grupo A} & $x_i - \bar{x}$ & $|x_i - \bar{x}|$ & $(x_i -
 \bar{x})^2$ & $x_i^2$ \\ \hline
 3 & -2 & 2 & 4 & 9 \\
 4 & -1 & 1 & 1 & 16 \\
 5 & 0 & 0 & 0 & 25 \\
 6 & 1 & 1 & 1 & 36 \\
 7 & 2 & 2 & 4 & 49 \\
 \hline
 Soma & 0 & 6 & 10 & 135 \\
 \hline
 \end{tabular}
\end{table}

A variância é

\(var_{obs} = \frac{1}{n}\sum_{i=1}^n (x_i - \bar{x}_{obs})^2 =  \frac{10}{5} = 2.\)

Ou, usando a fórmula alternativa

\(var_{obs} = \frac{1}{n}\sum_{i=1}^n x_i^2 - \bar{x}_{obs}^2 =  \frac{135}{5} - 5^2 = 2.\)
\end{frame}

\setbeamercovered{transparent}
\begin{frame}
\frametitle{Coeficiente de variação}

O \textbf{coeficiente de variação} para um conjunto de dados é definido
por \[
cv_{obs} = \frac{dp_{obs}}{\bar{x}_{obs}}
\] É uma medida \textbf{adimensional}, e geralmente apresentada na forma
de porcentagem.

No exemplo anterior:
\(dp_{obs} = \sqrt{var_{obs}} = \sqrt{2} = 1,414214\).

Portanto: \[
cv_{obs} = \frac{dp_{obs}}{\bar{x}_{obs}} = \frac{1,414214}{5} = 0,2828427
\approx 28,3\%
\]
\end{frame}

\setbeamercovered{transparent}
\begin{frame}
\frametitle{Variância em tabelas de frequência}

Assim como no caso da média, se tivermos \(n\) observações da variável
\(X\), das quais \(n_1\) são iguais a \(x_1\), \(n_2\) são iguais a
\(x_2\), \ldots, \(n_k\) são iguais a \(x_k\), então a variância pode
ser definida por:

\[
var_{obs}(X) = \frac{1}{n} \sum_{i=1}^{k} n_i (x_i - \bar{x}_{obs})^2
\]

Ou, pela fórmula alternativa:

\[
var_{obs}(X) = \frac{1}{n} \sum_{i=1}^{k} n_i x_{i}^{2} -
\bar{x}_{obs}^2
\]
\end{frame}

\setbeamercovered{transparent}
\begin{frame}
\frametitle{Exemplo}

Como exemplo, considere a tabela de frequência abaixo
(\(\bar{x} = 1,65\)):

\begin{table}[h]
  \centering
  \small
  \begin{tabular}{ccccc}
  \hline
  \textbf{Número} & \textbf{$n_i$} & \textbf{$f_i$}
  & \textbf{$x_i - \bar{x}$} & \textbf{$(x_i - \bar{x})^2$} \\
  \hline
  0 & 4 & 0,20 & -1,65 & 2,72 \\
  1 & 5 & 0,25 & -0,65 & 0,42 \\
  2 & 7 & 0,35 & 0,35 & 0,12 \\
  3 & 3 & 0,15 & 1,35 & 1,82 \\
  5 & 1 & 0,05 & 3,35 & 11,22 \\
  \hline
  \textbf{Total} & 20 & 1 & &  \\
  \hline
  \end{tabular}
\end{table}

A variância pode ser calculada por: \begin{align*}
var_{obs} =& \frac{(4 \cdot 2,72 + 5 \cdot 0,42 + \cdots + 1 \cdot 11,22 ) } {4 + 5 + 7 + 3 + 1} \\
  =& \frac{30,55}{20} \\
  =& 1,528
\end{align*}
\end{frame}

\setbeamercovered{transparent}
\begin{frame}
\frametitle{Exemplo}

Considere a seguinte tabela de distribuição de frequência
(\(\bar{x} = 11,22\)):

\begin{table}[h]
 \centering
 \small
 \begin{tabular}{lccccc}
 \hline
 \textbf{Classe} & \textbf{PM = $x_i$} & \textbf{$n_i$}
 & \textbf{$f_i$} & \textbf{$x_i - \bar{x}$} & \textbf{$(x_i - \bar{x})^2$} \\
 \hline
 $[4,8)$ & 6 & 10 & 0,278 & -5,222 & 27,272 \\
 $[8,12)$ & 10 & 12 & 0,333 & -1,222 & 1,494 \\
 $[12,16)$ & 14 & 8 & 0,222 & 2,778 & 7,716 \\
 $[16,20)$ & 18 & 5 & 0,139 & 6,778 & 45,938 \\
 $[20,24)$ & 22 & 1 & 0,028 & 10, 778 & 116,160 \\
 \hline
 \textbf{Total} & & 36 & 1 & & \\
 \hline
 \end{tabular}
\end{table}

Considerando os \textbf{pontos médios} de cada classe como os valores
\(x_i\), a variância pode ser calculada por \begin{align*}
 var_{obs} &= \frac{( 10 \cdot 27,272 + 12 \cdot 1,494 +
           \cdots + 1 \cdot 116,160 ) } {10 + 12 + 8 + 5 + 1}\\
 &= \frac{698,22}{36} = 19,395
\end{align*}
\end{frame}

\setbeamercovered{transparent}
\begin{frame}
\frametitle{Exemplo 4.9}

No Exemplo \(4.3\), definimos a quantidade \(D\), despesa no vestibular,
obtida a partir de \(X\) pela expressão \(D = 50X + 1300\), com \(X\)
indicando o número de vestibulares prestados.

\begin{table}
\centering
\begin{tabular}{ll}
\hline
\multicolumn{1}{l}{X} & \multicolumn{1}{l}{$n_i$} \\ \hline
1                       & 75                        \\
2                       & 47                        \\
3                       & 21                        \\
4                       & 7                        \\ \hline
\end{tabular}
\end{table}

Calcule a variância de \(D\).


\end{frame}

\begin{frame}
\frametitle{Exemplo 4.9}
\begin{columns}[T]
\begin{column}{5cm}
\begin{table}
\centering
\begin{tabular}{ll}
\hline
\multicolumn{1}{l}{X} & \multicolumn{1}{l}{$n_i$} \\ \hline
1                       & 75                        \\
2                       & 47                        \\
3                       & 21                        \\
4                       & 7                        \\ \hline
\end{tabular}
\end{table}
\end{column}
\begin{column}{10cm}
\end{column}
\end{columns}

\end{frame}



\setbeamercovered{transparent}
\begin{frame}
\frametitle{Variância de uma VA discreta}

Calcula o valor esperado: \(\mu = E(X) = \sum_{i=1}^k x_i p_i\)

Multiplica o quadrado dos desvios em torno do valor esperado pela
probabilidade e soma

\[ Var(X) =  \sum_{i=1}^k (x_i - \mu)^2 p_i \]

Alternativamente, podemos usar

\[ Var(X) = E[(X - \mu)^2] = E(X^2) - E(X)^2 \]

com \(E(X^2) = \sum_{i=1}^k x_i^2 p_i\)

\begin{itemize}
\item
  Ver Tabelas resumo \(4.2\) e \(4.3\).
\end{itemize}
\end{frame}

\setbeamercovered{transparent}
\begin{frame}
\frametitle{Propriedades da média e variância}
\begin{table}
\centering
\begin{tabular}{cc}
\hline
Conjunto de Dados & Variável aleatório \\ 
\hline
\(Y = aX+b\)   & \(Y = aX+b\)  \\
\hline
\(\bar{y}_{obs}= a\bar{x}_{obs}+b\) & \(E(Y) = aE(X)+b\)    \\
\hline
\(var_{obs}= a^2var_{obs}(X)\) & \(Var(Y) = a^2Var(X)\)    \\
\hline
\end{tabular}
\end{table}


\end{frame}


\setbeamercovered{transparent}
\begin{frame}
\frametitle{Exemplo 4.11}

Uma pequena cirurgia dentária pode ser realizada por três métodos
diferentes cujos tempos de recuperação (em dias) são modelados pelas
variáveis \(X_1\), \(X_2\) e \(X_3\). Admita suas funções de
probabilidades são dadas por

\begin{table}
\centering
\begin{tabular}{l|lllll}
$X_1$ & 0  & 4  & 5  & 6 & 10  \\ \hline
$p_i$ & 0.2 & 0.2 & 0.2 & 0.2 & 0.2
\end{tabular}
\end{table}

\begin{table}
\centering
\begin{tabular}{l|lll}
$X_2$ & 1     & 5     & 9    \\ \hline
$p_i$ & $1/3$ & $1/3$ & $1/3$
\end{tabular}
\end{table}

\begin{table}
\centering
\begin{tabular}{l|lll}
$X_3$ & 4   & 5   & 6   \\ \hline
$p_i$ & 0.3 & 0.4 & 0.3
\end{tabular}
\end{table}

Calcule as medidas de posição central e dispersão para cada VA e decida
sobre o método mais eficiente.
\end{frame}

\setbeamercovered{transparent}
\begin{frame}
\frametitle{Exemplo 4.11}
\end{frame}

\setbeamercovered{transparent}
\begin{frame}
\frametitle{Esperança e variância de modelos teóricos}

\begin{itemize}
\item
  Exemplo \(4.14\): Seja \(X\) com distribuição Bernoulli de parâmetro
  \(p\). Calcule a esperança e a variância de \(X\).
\item
  Exemplo \(4.15\): Seja \(X\) com distribuição Binomial de parâmetros
  \(n\) e \(p\). Calcule a esperança e a variância de \(X\).
\item
  Ver resultados da Tabela \(4.4\).
\end{itemize}
\vspace{1in}
\vspace{1in}
\vspace{1in}
\end{frame}

\setbeamercovered{transparent}
\begin{frame}
\frametitle{Esperança e variância de modelos teóricos}
\begin{table}
\centering
\begin{tabular}{|c|c|c|}
\hline
Variável Discreta & Valor Esperado & Variância \\ 
\hline
$Uniforme(1, k)$ & $\frac{1+k}{2}$ & $\frac{k^2-1}{12}$ \\
\hline
$Benoulli(p)$ & $p$ & $p(1-p)$ \\
\hline
$Binomial(n, p)$ & $np$ & $np(1-p)$ \\
\hline
$Geométrica(p)$ & $\frac{1-p}{p}$ & $\frac{1-p}{p^2}$ \\
\hline
$Poisson(\lambda)$ & $\lambda$ & $\lambda$ \\
\hline
$Hipergeometrica(n,m,r)$ & $\frac{rm}{n}$ & $\frac{rm(n-m)(n-r)}{n^2(n-1)}$ \\
\hline
\end{tabular}
\end{table}


\end{frame}

\setbeamercovered{transparent}
\begin{frame}
\frametitle{Exercícios recomendados}

\begin{itemize}
\item
  Seção \(4.2\) - \(2,3,4\) e \(6\).
\item
  Seção \(4.3\) - \(2,3,4,5\) e \(6\).
\item
  Extras: Seção 4.4 - 7 e 15.
\end{itemize}
\end{frame}


\end{document}