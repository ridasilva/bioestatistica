\documentclass[11pt]{beamer}
\usetheme{CambridgeUS}
\usepackage[utf8]{inputenc}
\usepackage{amsmath}
\usepackage{amsfonts}
\usepackage{amssymb}
\usepackage[
backend=biber,
style=alphabetic,
citestyle=authoryear
]{biblatex}

% Footnote without number
\newcommand\blfootnote[1]{%
  \begingroup
  \renewcommand\thefootnote{}\footnote{#1}%
  \addtocounter{footnote}{-1}%
  \endgroup
}

\addbibresource{stats.bib}
\title[Bioestatística II] %optional
{Variáveis Bidimensionais}

\subtitle{CGF2046 - Bioestatística II}

\author[da Silva, Ricardo] % (optional, for multiple authors)
{R. ~R. ~da Silva\inst{1}}

\institute[FCFRP] % (optional)
{
  \inst{1}%
  Departamento de Ciências BioMoleculares\\
  Faculdade de Ciências Farmacêuticas

}

\date{\today} % (optional)

\titlegraphic{\includegraphics[width=5.8cm]{figs/logo_final}} 

\begin{document}

\begin{frame}
\titlepage
\end{frame}

\begin{frame}
\label{contents}
\frametitle{Sumário}
\tableofcontents
\end{frame}

\section{Introdução}
\setbeamercovered{transparent}
\begin{frame}
\frametitle{Introdução}

\begin{itemize}
\item
  Interesse no comportamento conjunto de várias variáveis.
\item
  Construção de tabelas de frequência conjunta ou função de
  probabilidade conjunta.
\item
  O principal objetivo é explorar relações (similaridades) entre as
  colunas (ou linhas).
\item
  Determinar se existe \textbf{associação} entre as variáveis.
\item
  Podemos ter três situações:

  \begin{enumerate}
  \def\labelenumi{\alph{enumi}.}
  \item
    Duas variáveis qualitativas
  \item
    Duas variáveis quantitativas
  \item
    Uma variável qualitativa e outra quantitativa
  \end{enumerate}
\end{itemize}

\end{frame}

\setbeamercovered{transparent}
\begin{frame}
\frametitle{Introdução}

\begin{itemize}
\item
  Em todas as situações o objetivo é encontrar as possíveis
  \textbf{relações} ou \textbf{associações} entre as duas variáveis
\item
  Essas relações podem ser detectadas por meio de \textbf{métodos
  gráficos} ou \textbf{medidas numéricas}
\item
  Para efeitos práticos: existe associação se existe uma
  \textbf{mudança} no comportamento de uma variável na presença de outra
\item
  Exemplo:

  \begin{enumerate}
  \def\labelenumi{\alph{enumi}.}
  \item
    Frequência esperada de pessoas com mais de 170 cm de altura
  \item
    Frequência esperada de pessoas com mais de 170 cm de altura por sexo
  \end{enumerate}
\item
  Se a resposta for a mesma, dizemos que não há associação
\end{itemize}
\end{frame}

\section{Distribuições conjuntas e marginais}
\setbeamercovered{transparent}
\begin{frame}
\frametitle{Exemplo 5.1}

Uma amostra de \(20\) alunos do primeiro ano de uma faculdade foi
escolhida. Perguntou-se aos alunos se \emph{trabalhavam}, variável que
foi representada por \(X\), e o \emph{número de vestibulares prestados},
variável representada por \(Y\). Os dados obtidos estão na tabela
abaixo.
\begin{table}[h]
\centering
    \small
    \begin{tabular}{ccccccccccc}
      \hline
X & não & sim & não & não & não & sim & sim & não & sim & sim \\
Y & 1 & 1 & 2 & 1 & 1 & 2 & 3 & 1 & 1 & 1\\
      \hline
    \end{tabular}
\end{table}


\begin{table}[h]
\centering
    \small
    \begin{tabular}{ccccccccccc}
      \hline
X & não & não & sim & não & sim & não & não & não & sim & não\\
Y & 2 & 2 & 1 & 3 & 2 & 2 & 2 & 1 & 3 & 2\\
      \hline
    \end{tabular}
\end{table}

\end{frame}

\setbeamercovered{transparent}
\begin{frame}
\frametitle{Exemplo 5.1}

Distribuição conjunta

\begin{table}[h]
\centering
    \small
    \begin{tabular}{cc}
    \hline
\texttt{(X,Y)} & Freq\\
      \hline
não,1 & 5\\
não,2 & 6\\
não,3 & 1\\
sim,1 & 4\\
sim,2 & 2\\
sim,3 & 2\\
Sum & 20\\

      \hline
    \end{tabular}
\end{table}
\end{frame}

\setbeamercovered{transparent}
\begin{frame}
\frametitle{Exemplo 5.1}

Distribuição conjunta: \textbf{tabela de dupla entrada} (melhor para
visualizar)

\begin{table}[h]
\centering
    \small
    \begin{tabular}{ccccc}
    \hline
X/Y & 1 & 2 & 3 & Sum\\
    \hline
não & 5 & 6 & 1 & 12\\
sim & 4 & 2 & 2 & 8\\
Sum & 9 & 8 & 3 & 20\\

      \hline
    \end{tabular}
\end{table}

Distribuição \textbf{marginal} de \(X\)

\begin{table}[h]
\centering
    \small
    \begin{tabular}{ccc}
    \hline
não & sim & Sum\\
    \hline
12 & 8 & 20\\
      \hline
    \end{tabular}
\end{table}

Distribuição \textbf{marginal} de \(Y\)

\begin{table}[h]
\centering
    \small
    \begin{tabular}{cccc}
    \hline
1 & 2 & 3 & Sum\\
    \hline
9 & 8 & 3 & 20\\
      \hline
    \end{tabular}
\end{table}
\end{frame}

\setbeamercovered{transparent}
\begin{frame}
\frametitle{Exemplo 5.2}

Um estudo envolveu \(345\) pacientes HIV positivos, acompanhados durante
um ano, pelo setor de doenças infecciosas de um grande hospital público.
Os dados apresentados contêm as ocorrências relacionadas às variáveis
\emph{número de internações} \((I)\) e \emph{número de crises com
infecções oportunistas} \((C)\).

\begin{table}[]
\centering
\begin{tabular}{l|lllll} \hline
I/C & 0  & 1  & 2  & 3  & 4  \\ \hline
0   & 84 & 21 & 8  & 2  & 0  \\
1   & 20 & 59 & 35 & 14 & 2  \\
2   & 6  & 11 & 43 & 28 & 12 \\ \hline
\end{tabular}
\end{table}

\begin{itemize}
\item
  Obtenha as marginais de \(I\) e \(C\).
\end{itemize}
\end{frame}

\setbeamercovered{transparent}
\begin{frame}
\frametitle{Exemplo 5.2}

\begin{itemize}
\item
  Marginal de \(I\) (soma das linhas)
\end{itemize}

\begin{table}[h]
\centering
    \small
    \begin{tabular}{cccc}
    \hline
0 & 1 & 2 & Sum\\
    \hline
115 & 130 & 100 & 345\\
      \hline
    \end{tabular}
\end{table}


\begin{itemize}
\item
  Marginal de \(C\) (soma das colunas)
\end{itemize}

\begin{table}[h]
\centering
    \small
    \begin{tabular}{cccccc}
    \hline
0 & 1 & 2 & 3 & 4 & Sum\\
    \hline
110 & 91 & 86 & 44 & 14 & 345\\
      \hline
    \end{tabular}
\end{table}
\end{frame}

\setbeamercovered{transparent}
\begin{frame}
\frametitle{VAs discretas: função de probabilidade conjunta}

Sejam \(X\) e \(Y\) \textbf{duas VAs discretas} originárias do mesmo
fenômeno aleatório, com valores atribuídos a partir do mesmo espaço
amostral.

A \textbf{função de probabilidade conjunta} é definida, para todos os
possíveis pares de valores \((X,Y)\), da seguinte forma:

\[ p(x,y) = P[(X=x)\cap(Y=y)] = P(X=x,Y=y).\]

Ou seja, \(p(x,y)\) representa a probabilidade de \((X,Y)\) ser igual a
\((x,y)\).

A função de probabilidade conjunta também pode ser chamada de
\textbf{distribuição conjunta} ou simplesmente \textbf{conjunta} das
variáveis.
\end{frame}

\setbeamercovered{transparent}
\begin{frame}
\frametitle{Exemplo 5.4}

Uma empresa atende encomendas de supermercados dividindo os pedidos em
duas partes de modo a serem atendidos, de forma independente, pelas suas
duas fábricas. Devido à grande demanda, pode haver atraso no cronograma
de entrega, sendo que a fábrica \(I\) atrasa com probabilidade \(0.1\) e
a \(II\) com \(0.2\). Sejam \(A_I\) e \(A_{II}\) os eventos
correspondentes a ocorrência de atraso nas fábricas \(I\) e \(II\),
respectivamente.

Para uma entrega, a indústria recebe \(200\) u.m, mas paga \(20\) para
cada fábrica que atrasar. Considere que o supermercado que recebe a
encomenda fez um índice relacionado à pontualidade de entrega. Este
índice, atribuiu \(10\) pontos para cada entrega dentro do cronograma
previsto. Denote por \(X\) o valor recebido pelo pedido e \(Y\) o índice
obtido. Obtenha a conjunta de \(Y\) e \(X\) e as marginais de \(Y\) e
\(X\).
\end{frame}

\setbeamercovered{transparent}
\begin{frame}
\frametitle{Exemplo 5.4}
\end{frame}


\setbeamercovered{transparent}
\begin{frame}
\frametitle{Exemplo 5.5}

Uma região foi dividida em \(10\) sub-regiões. Em cada uma delas, foram
observadas duas variáveis: \emph{número de poços artesianos} \((X)\) e
\emph{número de riachos ou rios presentes} na sub-região \((Y)\). Os
resultados são apresentados na tabela a seguir:

\begin{table}[h]
\centering
    \small
    \begin{tabular}{ccccccccccc}
    \hline
Sub-região & 1 & 2 & 3 & 4 & 5 & 6 & 7 & 8 & 9 & 10\\
X & 0 & 0 & 0 & 0 & 1 & 2 & 1 & 2 & 2 & 0\\
Y & 1 & 2 & 1 & 0 & 1 & 0 & 0 & 1 & 2 & 2\\
      \hline
    \end{tabular}
\end{table}

\begin{itemize}
\item
  Construa a distribuição conjunta e marginais de \(X\) e \(Y\).
\end{itemize}
\end{frame}

\setbeamercovered{transparent}
\begin{frame}
\frametitle{Exemplo 5.5}

Consideramos que cada região tem a mesma probabilidade \(1/10\) de ser
escolhida. Assim a distribuição conjunta é:

\begin{table}[h]
\centering
    \small
    \begin{tabular}{cc}
    \hline
\texttt{(X,Y)} & \texttt{p(x,y)}\\
    \hline
0,0 & 0.1\\
0,1 & 0.2\\
0,2 & 0.2\\
1,0 & 0.1\\
1,1 & 0.1\\
2,0 & 0.1\\
2,1 & 0.1\\
2,2 & 0.1\\
Sum & 1.0\\
      \hline
    \end{tabular}
\end{table}

\end{frame}

\setbeamercovered{transparent}
\begin{frame}
\frametitle{Exemplo 5.5}

Uma forma mais conveniente é

\begin{table}[h]
\centering
    \small
    \begin{tabular}{cccc}
    \hline
X/Y & 0 & 1 & 2\\
    \hline
0 & 0.1 & 0.2 & 0.2\\
1 & 0.1 & 0.1 & 0.0\\
2 & 0.1 & 0.1 & 0.1\\
      \hline
    \end{tabular}
\end{table}

Para obter as marginais, efetuamos a soma nas linhas para obter a
marginal de \(X\), e nas colunas para obter a marginal de \(Y\). Por
exemplo, \(P(X = 0)\) é obtida através de: \begin{align*}
P(X = 0) &= P(X=0,Y=0) + P(X=0,Y=1) + P(X=0,Y=2) \\
 &= 0.1 + 0.2 + 0.2 = 0.5
\end{align*}
\end{frame}

\setbeamercovered{transparent}
\begin{frame}
\frametitle{Exemplo 5.5}

Repetindo os cálculos para todos os valores de \(X\) e \(Y\), obtemos as marginais:

\begin{table}[h]
\centering
    \small
    \begin{tabular}{ccccc}
    \hline
X/Y & 0 & 1 & 2 & P(X=x)\\
    \hline
0 & 0.1 & 0.2 & 0.2 & 0.5\\
1 & 0.1 & 0.1 & 0.0 & 0.2\\
2 & 0.1 & 0.1 & 0.1 & 0.3\\
P(Y=y) & 0.3 & 0.4 & 0.3 & 1.0\\
      \hline
    \end{tabular}
\end{table}

Marginal de \(X\)

\begin{table}[h]
\centering
    \small
    \begin{tabular}{cccc}
    \hline
0 & 1 & 2 & Sum\\
    \hline
0.5 & 0.2 & 0.3 & 1\\
      \hline
    \end{tabular}
\end{table}

Marginal de \(Y\)

\begin{table}[h]
\centering
    \small
    \begin{tabular}{cccc}
    \hline
0 & 1 & 2 & Sum\\
    \hline
0.3 & 0.4 & 0.3 & 1\\
      \hline
    \end{tabular}
\end{table}

\end{frame}

\setbeamercovered{transparent}
\begin{frame}
\frametitle{Funções de probabilidade marginal}

Da função de probabilidade conjunta \(p(x,y)\), é possível então obter
as \textbf{funções de probabilidade marginais} de \(X\) e \(Y\), através da soma de uma das coordenadas:

\[
P(X = x) = \sum_{y} p(x,y) \quad \text{e} \quad
P(Y = y) = \sum_{x} p(x,y)
\]

com o somatório percorrendo todos os valores de \(X\) ou \(Y\), conforme for o caso.
\end{frame}

\setbeamercovered{transparent}
\begin{frame}
\frametitle{Exemplo 5.6}

Em uma cidade, admite-se que o \emph{número de anos para competar o
ensino fundamental} (\(F\)), e o \emph{número de anos para completar o
ensino médio} (\(M\)) têm função de probabilidade conjunta:

\begin{columns}[T]
\begin{column}{7cm}

\begin{table}[h]
\centering
    \small
    \begin{tabular}{cc}
    \hline
\texttt{(F,M)} & \texttt{p(f,m)}\\
    \hline
9,3 & 0.30\\
9,4 & 0.10\\
9,5 & 0.10\\
10,3 & 0.20\\
10,4 & 0.05\\
10,5 & 0.10\\
11,4 & 0.10\\
11,5 & 0.05\\
      \hline
    \end{tabular}
\end{table}

\end{column}
\begin{column}{7cm}

\begin{table}[h]
\centering
    \small
    \begin{tabular}{cccc}
    \hline
F/M & 3 & 4 & 5\\
    \hline
9 & 0.3 & 0.10 & 0.10\\
10 & 0.2 & 0.05 & 0.10\\
11 & 0.0 & 0.10 & 0.05\\
      \hline
    \end{tabular}
\end{table}

\end{column}
\end{columns}

O interesse está em estudar o comportamento da variável \emph{número
total de anos para completar o ensino fundamental e médio}, ou seja,
\(T = F + M\). Obtenha a distribuição de \(T\).
\end{frame}

\setbeamercovered{transparent}
\begin{frame}
\frametitle{Exemplo 5.6}

\begin{columns}[T]
\begin{column}{7cm}

Possíveis valores de \(T = F+M\)

\begin{table}[h]
\centering
    \small
    \begin{tabular}{ccc}
    \hline
\texttt{(F,M)} & \texttt{p(f,m)} & T\\
    \hline
9,3 & 0.30 & 12\\
9,4 & 0.10 & 13\\
9,5 & 0.10 & 14\\
10,3 & 0.20 & 13\\
10,4 & 0.05 & 14\\
10,5 & 0.10 & 15\\
11,4 & 0.10 & 15\\
11,5 & 0.05 & 16\\
      \hline
    \end{tabular}
\end{table}

\end{column}
\begin{column}{7cm}

Distribuição de \(T=F+M\)

\begin{table}[h]
\centering
    \small
    \begin{tabular}{cc}
    \hline
T & P(T)\\
    \hline
12 & 0.30\\
13 & 0.30\\
14 & 0.15\\
15 & 0.20\\
16 & 0.05\\
      \hline
    \end{tabular}
\end{table}

\end{column}
\end{columns}
\end{frame}

\section{Associação entre variáveis}
\setbeamercovered{transparent}
\begin{frame}
\frametitle{Associação entre variáveis}

\begin{itemize}
\item
  Um dos principais objetivos de se construir uma distribuição conjunta
  de duas variáveis, é descrever a \textbf{associação} entre elas
\item
  Queremos conhecer o grau de \textbf{dependência}, para prever melhor o
  resultado de uma delas quando conhecemos a outra
\item
  Veremos algumas formas de medir/avaliar essa dependência:

  \begin{enumerate}
  \def\labelenumi{\alph{enumi}.}
  \item
    Duas variáveis quantitativas

    \begin{itemize}
    \item
      Diagramas de dispersão
    \item
      Probabilidades condicionais
    \item
      Correlação e covariância
    \end{itemize}
  \item
    Duas variáveis qualitativas

    \begin{itemize}
    \item
      Verificação de proporções através da distribuição conjunta
    \item
      Medida \(Q^2\)
    \end{itemize}
  \end{enumerate}
\end{itemize}
\end{frame}

\setbeamercovered{transparent}
\begin{frame}
\frametitle{Exemplo 5.7}

\begin{itemize}
\item
  Dentre os alunos do \(1^{\circ}\) ano do ensino médio de uma certa
  escola, selecionou-se os quinze alunos com melhor desempenho, (nota
  acima de \(7\)) em inglês. Para esses alunos, foi construída a tabela
  abaixo com as notas de inglês \((I)\), português \((P)\) e matemática
  \((M)\):
\end{itemize}


\begin{table}[h]
\centering
    \small
    \begin{tabular}{cccccccccccccccc}
    \hline
I & 7 & 7 & 7 & 7 & 8 & 8 & 8 & 8 & 8 & 8 & 8 & 9 & 9 & 9 &
10\\
P & 8 & 6 & 8 & 9 & 8 & 6 & 9 & 7 & 7 & 6 & 7 & 8 & 9 & 8 &
8\\
M & 5 & 6 & 7 & 5 & 5 & 5 & 6 & 4 & 7 & 6 & 5 & 5 & 6 & 5 &
5\\
      \hline
    \end{tabular}
\end{table}

\begin{itemize}
\item
  Obtenha as distribuições conjuntas e gráficos de dispersão.
\end{itemize}
\end{frame}

\setbeamercovered{transparent}
\begin{frame}
\frametitle{Exemplo 5.7 - Distribuições conjuntas}

\begin{columns}[T]
\begin{column}{7cm}

Inglês e Português:

\begin{table}[h]
\centering
    \small
    \begin{tabular}{ccccc}
    \hline
I/P & 6 & 7 & 8 & 9\\
   \hline
7 & 1 & 0 & 2 & 1\\
8 & 2 & 3 & 1 & 1\\
9 & 0 & 0 & 2 & 1\\
10 & 0 & 0 & 1 & 0\\
    \hline
    \end{tabular}
\end{table}

Inglês e Matemática:

\begin{table}[h]
\centering
    \small
    \begin{tabular}{ccccc}
    \hline
I/M & 4 & 5 & 6 & 7\\
   \hline
7 & 0 & 2 & 1 & 1\\
8 & 1 & 3 & 2 & 1\\
9 & 0 & 2 & 1 & 0\\
10 & 0 & 1 & 0 & 0\\
    \hline
    \end{tabular}
\end{table}

\end{column}
\begin{column}{7cm}

Português e Matemática:

\begin{table}[h]
\centering
    \small
    \begin{tabular}{ccccc}
    \hline
P/M & 4 & 5 & 6 & 7\\
    \hline
6 & 0 & 1 & 2 & 0\\
7 & 1 & 1 & 0 & 1\\
8 & 0 & 5 & 0 & 1\\
9 & 0 & 1 & 2 & 0\\
      \hline
    \end{tabular}
\end{table}

\end{column}
\end{columns}
\end{frame}

\setbeamercovered{transparent}
\begin{frame}
\frametitle{Exemplo 5.7 - Diagramas de dispersão}

\begin{center}\includegraphics[width=1\linewidth]{figs/unnamed-chunk-25-1} \end{center}
\end{frame}

\setbeamercovered{transparent}
\begin{frame}
\frametitle{Probabilidade condicional para VAs discretas}

\begin{itemize}
\item
  A \textbf{probabilidade condicional} de \(X = x\), dado que \(Y = y\)
  ocorreu, é dada pela expressão: \[
  P(X = x | Y = y) = \frac{P(X=x,Y=y)}{P(Y=y)},
  \quad \text{se} \quad P(Y=y) > 0.
  \]
\item
  Duas VAs discretas são \textbf{independentes}, se a ocorrência de
  qualquer valor de uma delas não altera a probabilidade de valores da
  outra. Em termos matemáticos \[ P(X = x | Y = y) = P(X = x).\]
\item
  Definição alternativa
  \[ P(X =x, Y = y) = P(X=x)P(Y=y), \quad \forall (x,y).\]
\end{itemize}
\end{frame}

\setbeamercovered{transparent}
\begin{frame}
\frametitle{Exemplo 5.8}

O Centro Acadêmico de uma faculdade de administração fez um levantamento
da \emph{remuneração dos estágios dos alunos}, em salários mínimos
(\(X\)), com relação ao \emph{ano que estão cursando} (\(Y\)). As
probabilidades de cada caso são apresentadas na próxima tabela,
incluindo as distribuições marginais.

\begin{table}
\centering
\begin{tabular}{l|llll|l} \hline
Salario/Ano & 2    & 3    & 4    & 5    & $P(Sal = x)$ \\ \hline
2             & $2/25$ & $2/25$ & $1/25$ & $0$    & $5/25$       \\
3             & $2/25$ & $5/25$ & $2/25$ & $2/25$ & $11/25$      \\
4             & $1/25$ & $2/25$ & $2/25$ & $4/25$ & $9/25$       \\ \hline
$P(Ano = y)$    & $5/25$ & $9/25$ & $5/25$ & $6/25$ & $1$         \\ \hline
\end{tabular}
\end{table}

\begin{itemize}
\item
  \(X\) e \(Y\) são independentes?
\end{itemize}
\end{frame}

\setbeamercovered{transparent}
\begin{frame}
\frametitle{Exemplo 5.9}

Em uma clínica médica foram coletados dados em 150 pacientes, referentes
ao último ano. Observou-se a \emph{ocorrência de infecções urinárias}
(\(U\)) e o \emph{número de parceiros sexuais} (\(N\)).

\begin{table}
\centering
\begin{tabular}{l|lll|l} \hline
U/N & 0  & 1  & 2 + & Total \\ \hline
Sim   & 12 & 21 & 47  & 80    \\
Não   & 45 & 18 & 7   & 70    \\ \hline
Total & 57 & 39 & 54  & 150  \\ \hline
\end{tabular}
\end{table}

\begin{itemize}
\item
  Estude a associação entre \(U\) e \(N\).
\end{itemize}
\end{frame}

\setbeamercovered{transparent}
\begin{frame}
\frametitle{Exemplo 5.9}

\begin{itemize}
\item
  Ao invés de trabalharmos com as frequências absolutas, podemos
  construir tabelas com as frequências relativas, mas aqui existem três
  possibilidades para expressar as proporções:

  \begin{enumerate}
  \def\labelenumi{\alph{enumi}.}
  \item
    em relação ao total geral
  \item
    em relação ao total de cada linha
  \item
    em relação ao total de cada coluna
  \end{enumerate}
\item
  A escolha depende do objetivo do estudo, mas não altera a conclusão
\end{itemize}
\end{frame}

\setbeamercovered{transparent}
\begin{frame}
\frametitle{Exemplo 5.9}

\begin{itemize}
\item
  Tabela com porcentagens em relação ao total de coluna.
\end{itemize}

\begin{table}
\centering
\begin{tabular}{l|lll|l} \hline
U/N & 0  & 1  & 2 + & Total \\ \hline
Sim   & 21,1\% & 53,8\% & 87,0\%  & 53,3\%    \\
Não   & 78,9\% & 46,2\% & 13,0\%  & 46,7\%    \\ \hline
Total & 100\% & 100\% & 100\%  & 100\%  \\ \hline
\end{tabular}
\end{table}

\begin{itemize}
\item
  Independente de \(N\), a porcentagem de pessoas com infecção é 53,3\%
  (46,7\% sem infecção).
\item
  Caso não exista associação de \(U\) com \(N\), deveríamos esperar
  porcentagens similares em cada valor de \(N\) (independência).
\item
  Analisar os percentuais em relação ao total das linhas levaria à mesma
  conclusão.
\end{itemize}
\end{frame}

\setbeamercovered{transparent}
\begin{frame}
\frametitle{Exemplo 5.10}

Os dados abaixo representam uma amostra de 80 famílias de um certo
bairro, onde \(T\) é o \emph{número de pessoas que trabalham na
família}, e \(A\) é o \emph{número de adolescentes entre 12 e 18 anos}.

\begin{table}[h]
\centering
    \small
    \begin{tabular}{ccccccc}
    \hline
T/A & 0 & 1 & 2 & 3 & 4 & Sum\\
    \hline
0 & 5 & 4 & 2 & 3 & 1 & 15\\
1 & 2 & 8 & 6 & 4 & 1 & 21\\
2 & 4 & 8 & 8 & 5 & 2 & 27\\
3 & 4 & 2 & 2 & 5 & 4 & 17\\
Sum & 15 & 22 & 18 & 17 & 8 & 80\\
      \hline
    \end{tabular}
\end{table}

\begin{itemize}
\item
  Verifique a associação entre as duas variáveis.
\end{itemize}
\end{frame}

\setbeamercovered{transparent}
\begin{frame}
\frametitle{Exemplo 5.10}

Assim como no exemplo anterior, podemos usar os totais de coluna e
calcular as frequências relativas

\begin{table}[h]
\centering
    \small
    \begin{tabular}{ccccccc}
    \hline
T/A & 0 & 1 & 2 & 3 & 4 & Sum\\
    \hline
0 & 0.33 & 0.18 & 0.11 & 0.18 & 0.12 & 0.19\\
1 & 0.13 & 0.36 & 0.33 & 0.24 & 0.12 & 0.26\\
2 & 0.27 & 0.36 & 0.44 & 0.29 & 0.25 & 0.34\\
3 & 0.27 & 0.09 & 0.11 & 0.29 & 0.50 & 0.21\\
Sum & 1.00 & 1.00 & 1.00 & 1.00 & 1.00 & 1.00\\
      \hline
    \end{tabular}
\end{table}

\begin{itemize}
\item
  Caso houvesse independência, as frequências de cada valor de \(A\)
  deveriam ser próximas das frequências marginais de \(T\) (última
  coluna).
\item
  Podemos também verificar se as \textbf{frequências absolutas} de cada
  coluna de \(A\) se distribuem de maneira \textbf{proporcional}.
\item
  Dessa forma, calculamos as \textbf{frequências esperadas}, caso
  houvesse independência.
\end{itemize}
\end{frame}

\setbeamercovered{transparent}
\begin{frame}
\frametitle{Exemplo 5.10}

\begin{columns}[T]
\begin{column}{7cm}

Distribuição marginal de \(T\)

\begin{table}[h]
\centering
    \small
    \begin{tabular}{ccc}
    \hline
T & Freq & Freq. rel.\\
\hline
0 & 15 & 0.19\\
1 & 21 & 0.26\\
2 & 27 & 0.34\\
3 & 17 & 0.21\\
Sum & 80 & 1.00\\
    \hline
    \end{tabular}
\end{table}

\begin{align*}
\text{Freq. esp.} &= \text{Freq. rel.} \times \text{Total de col.} \\
2.85 &= 0.19 \times 15 \\
3.90 &= 0.26 \times 15 \\
5.10 &= 0.34 \times 15 \\
3.15 &= 0.21 \times 15
\end{align*}

\end{column}
\begin{column}{7cm}

Distribuição de \(T/A=0\)

\begin{table}[h]
\centering
    \small
    \begin{tabular}{cc}
    \hline
\texttt{T/A=0} & Freq. obs. \\
    \hline
0 & 5\\
1 & 2\\
2 & 4\\
3 & 4\\
Sum & 15\\
      \hline
    \end{tabular}
\end{table}


\begin{table}[h]
\centering
    \small
    \begin{tabular}{ccc}
    \hline
\texttt{T/A=0} & Freq. obs. & Freq. esp.\\
    \hline
0 & 5 & 2.85\\
1 & 2 & 3.90\\
2 & 4 & 5.10\\
3 & 4 & 3.15\\
Sum & 15 & 15.00\\
      \hline
    \end{tabular}
\end{table}

\end{column}
\end{columns}
\end{frame}

\setbeamercovered{transparent}
\begin{frame}
\frametitle{Exemplo 5.10}

Fazendo o mesmo processo para todas as colunas, obtemos a tabela com as
\textbf{frequências esperadas}

\begin{table}[h]
\centering
    \small
    \begin{tabular}{ccccccc}
    \hline
T/A & 0 & 1 & 2 & 3 & 4 & Sum\\
    \hline
0 & 2.81 & 4.12 & 3.38 & 3.19 & 1.5 & 15\\
1 & 3.94 & 5.78 & 4.73 & 4.46 & 2.1 & 21\\
2 & 5.06 & 7.43 & 6.08 & 5.74 & 2.7 & 27\\
3 & 3.19 & 4.67 & 3.82 & 3.61 & 1.7 & 17\\
Sum & 15.00 & 22.00 & 18.00 & 17.00 & 8.0 & 80\\
      \hline
    \end{tabular}
\end{table}

que podemos comparar com a tabela de \textbf{frequências observadas}

\begin{table}[h]
\centering
    \small
    \begin{tabular}{ccccccc}
    \hline
T/A & 0 & 1 & 2 & 3 & 4 & Sum\\
    \hline
0 & 5 & 4 & 2 & 3 & 1 & 15\\
1 & 2 & 8 & 6 & 4 & 1 & 21\\
2 & 4 & 8 & 8 & 5 & 2 & 27\\
3 & 4 & 2 & 2 & 5 & 4 & 17\\
Sum & 15 & 22 & 18 & 17 & 8 & 80\\
      \hline
    \end{tabular}
\end{table}
\end{frame}

\setbeamercovered{transparent}
\begin{frame}
\frametitle{Exemplo 5.10}

Agora podemos quantificar as diferenças entre as \textbf{frequências
observadas} (\(o_{ij}\)), e as \textbf{frequências esperadas}
(\(e_{ij}\)) através de \[
Q^2 = \sum_{i,j} \frac{(o_{ij} - e_{ij})^2}{e_{ij}}
\] Dessa forma, temos: \[
Q^2 = \frac{(5 - 2.81)^2}{2.81} + \cdots + \frac{(4 - 1.7)^2}{1.7} = 12.63
\] Se as frequências esperadas fossem muito próximas das observadas,
esperaríamos que esse valor fosse próximo de zero.

Como o valor é relativamente alto, há uma indicação de que as duas
variáveis são dependentes.
\end{frame}

\setbeamercovered{transparent}
\begin{frame}
\frametitle{Correlação entre variáveis num conjunto de dados}

\begin{itemize}
\item
  Considere um conjunto de dados com \(n\) pares de valores para as
  variáveis \(X\) e \(Y\). O coeficiente de correlação mede a
  dependência linear entre as variáveis e é calculado por
  \[ \rho_{XY} = \frac{\sum_{i=1}^n(x_i - \bar{x})(y_i - \bar{y})}{\sqrt{[\sum_{j=1}^n(x_j - \bar{x})^2][\sum_{j=1}^n(y_j - \bar{y})^2] ]}}.\]
\item
  Formula mais conveniente para cálculos
  \[ \rho_{XY} = \frac{\sum_{i=1}^n x_i y_i - n\bar{x}\bar{y}}{\sqrt{[\sum_{j=1}^n x_j^2 - n \bar{x}^2] [\sum_{j=1}^n y_j^2 - n \bar{y}^2]  }}.\]
\item
  Note que \(-1 \leq \rho_{XY} \leq 1\).
\item
  Observação: \(\rho_{XY} = 0\) não indica independência.
\end{itemize}
\end{frame}

\setbeamercovered{transparent}
\begin{frame}
\frametitle{Exemplo 5.11}

A quantidade de chuva é um fator importante na produtividade agrícola.
Para medir esse efeito foram anotados, para \(8\) diferentes regiões
produtoras de soja, o \emph{índice pluviométrico} em milímetros \((X)\)
e a \emph{produção do último ano} em toneladas \((Y)\).

\begin{table}[h]
\centering
    \small
    \begin{tabular}{ccccccccc}
    \hline
X & 120 & 140 & 122 & 150 & 115 & 190 & 130 & 118\\
Y & 40 & 46 & 45 & 37 & 25 & 54 & 33 & 30\\
      \hline
    \end{tabular}
\end{table}

\begin{itemize}
\item
  Determine o coeficiente de correlação.
\end{itemize}
\end{frame}

\setbeamercovered{transparent}
\begin{frame}
\frametitle{Exemplo 5.11}

\begin{center}\includegraphics[width=0.6\linewidth]{figs/unnamed-chunk-34-1} \end{center}


\[
\rho_{XY} = \frac{43245 - 8 \times 135.63 \times 38.75}
  {\sqrt{[151533 - 8 \times 135.63^2] [12640 - 8 \times 38.75^2]}}
  = 0,73
\]
\end{frame}

\setbeamercovered{transparent}
\begin{frame}
\frametitle{Propriedades de esperança de VAs}

Para podermos definir medidas de dependência entre \textbf{VAs
discretas}, precisamos das seguintes propriedades de esperança de VAs.

\begin{block}{Valor esperado da soma de VAs}
Para variáveis aleatórias \(X\) e \(Y\), vale sempre que \[
E(X + Y) = E(X) + E(Y)
\]
\end{block}
\end{frame}

\setbeamercovered{transparent}
\begin{frame}
\frametitle{Voltando ao exemplo 5.6}


\begin{columns}[T]
\begin{column}{7cm}

\begin{table}[h]
\centering
    \small
    \begin{tabular}{ccccc}
    \hline
F/M & 3 & 4 & 5 & Sum\\
\hline
9 & 0.3 & 0.10 & 0.10 & 0.50\\
10 & 0.2 & 0.05 & 0.10 & 0.35\\
11 & 0.0 & 0.10 & 0.05 & 0.15\\
Sum & 0.5 & 0.25 & 0.25 & 1.00\\
    \hline
    \end{tabular}
\end{table}

\(E(F) = 9 \times 0.5 + \cdots + 11 \times 0.15 = 9.65\)
\(E(M) = 3 \times 0.5 + \cdots + 5 \times 0.25 = 3.75\)


\end{column}
\begin{column}{7cm}

Distribuição de \(T=F+M\)

\begin{table}[h]
\centering
    \small
    \begin{tabular}{cccccc}
    \hline
T & 12.0 & 13.0 & 14.00 & 15.0 & 16.00\\
P(T) & 0.3 & 0.3 & 0.15 & 0.2 & 0.05\\
      \hline
    \end{tabular}
\end{table}

\end{column}
\end{columns}

\begin{align*}
E(F+M) &= 12 \times 0.3 + \cdots + 16 \times 0.05 =
13.4 \\
 \text{ou} & \\
 E(F+M) &= E(F) + E(M) \\
 &= 9.75 + 3.75 \\
 &= 13.4
\end{align*}
\end{frame}

\setbeamercovered{transparent}
\begin{frame}
\frametitle{Propriedades de esperança de VAs}

\begin{block}{Valor esperado do produto de VAs}
Para \(X\) e \(Y\) variáveis aleatórias discretas
\textbf{independentes}, temos \[
E(XY) = E(X)E(Y)
\]
\end{block}

\begin{block}{IMPORTANTE}
\(X\) e \(Y\) independentes \(\Rightarrow\) \(E(XY) = E(X)E(Y)\)
\newline

No entanto: \newline

\(E(XY) = E(X)E(Y)\) \(\not\Rightarrow\) \(X\) e \(Y\) independentes
\end{block}
\end{frame}

\setbeamercovered{transparent}
\begin{frame}
\frametitle{Exemplo 5.13}

Considere as variáveis \(W\) e \(Z\) com a seguinte distribuição:

\begin{columns}[T]
\begin{column}{7cm}

\begin{table}[h]
\centering
    \small
    \begin{tabular}{ccccc}
    \hline
W/Z & 2 & 3 & 4 & Sum\\
\hline
-1 & 0.17 & 0.00 & 0.25 & 0.42\\
0 & 0.00 & 0.08 & 0.08 & 0.17\\
1 & 0.08 & 0.17 & 0.17 & 0.42\\
Sum & 0.25 & 0.25 & 0.50 & 1.00\\
    \hline
    \end{tabular}
\end{table}

\end{column}
\begin{column}{7cm}
\(E(W) = -1 \times 0.42 + \cdots + 1 \times 0.42 = 0\)
\(E(Z) = 2 \times 0.25 + \cdots + 4 \times 0.5 = 3.25\)

\end{column}
\end{columns}

A variável \(WZ\) terá então a distribuição 
\begin{table}[h]
\centering
    \small
    \begin{tabular}{ccccccc}
    \hline
WZ & -4.00 & -2.00 & 0.00 & 2.00 & 3.00 & 4.00\\
pWZ & 0.25 & 0.17 & 0.17 & 0.08 & 0.17 & 0.17\\
    \hline
    \end{tabular}
\end{table}

\(E(WZ) = -4 \times 0.25 + \cdots + 4 \times 0.17  = 0  = E(W)E(Z)\)

\textbf{Mas}, verifique que \(W\) e \(Z\) \textbf{não são independentes}!
\end{frame}

\setbeamercovered{transparent}
\begin{frame}
\frametitle{Covariância de duas VAs}

Uma medida de dependência linear entre \(X\) e \(Y\) é a covariância: \[
Cov(X, Y) = \sigma_{XY} = E[(X-\mu_X)(Y-\mu_y)].
\] Uma forma alternativa (mais fácil de calcular) é: \[
Cov(X, Y) = \sigma_{XY} = E(XY) - E(X)E(Y)
\]

\begin{block}{Variância da soma de duas VAs}
\[
Var(X + Y) = Var(X) + Var(Y) + 2 Cov(X,Y)
\]
\end{block}
\end{frame}

\setbeamercovered{transparent}
\begin{frame}
\frametitle{Exemplo 5.14}

\begin{itemize}
\item
  As variáveis \(U\) e \(V\) têm a seguinte distribuição conjunta.
\end{itemize}

\begin{table}
\centering
\begin{tabular}{l|lllll|l} \hline
U/V & 2   & 4   & 6   & 8   & 10  & $P(U=u)$   \\
\hline
2  & 0.1 & 0   & 0   & 0   & 0   & 0.1 \\
3  & 0   & 0.2 & 0   & 0.1 & 0   & 0.3 \\
4  & 0   & 0   & 0.2 & 0   & 0   & 0.2 \\
5  & 0   & 0.1 & 00  & 0.2 & 0   & 0.3 \\
6  & 0   & 0   & 0   & 0   & 0.1 & 0.1 \\ \hline
$P(V=v)$  & 0.1 & 0.3 & 0.2 & 0.3 & 0.1 & 1 \\
\hline
\end{tabular}
\end{table}

\begin{itemize}
\item
  Calcule a covariância entre \(U\) e \(V\).
\end{itemize}
\end{frame}

\setbeamercovered{transparent}
\begin{frame}
\frametitle{Exemplo 5.14}

\begin{columns}[T]
\begin{column}{7cm}

Marginais de \(U\), \(V\), e \(UV\):

\begin{table}[h]
\centering
    \small
    \begin{tabular}{cccccc}
\hline
U & 2.0 & 3.0 & 4.0 & 5.0 & 6.0\\
pU & 0.1 & 0.3 & 0.2 & 0.3 & 0.1\\
    \hline
    \end{tabular}
\end{table}

\begin{table}[h]
\centering
    \small
    \begin{tabular}{cccccc}
\hline
V & 2.0 & 4.0 & 6.0 & 8.0 & 10.0\\
pV & 0.1 & 0.3 & 0.2 & 0.3 & 0.1\\
    \hline
    \end{tabular}
\end{table}

\begin{table}[h]
\centering
    \small
    \begin{tabular}{ccccccc}
\hline
UV & 4.0 & 12.0 & 20.0 & 24.0 & 40.0 & 60.0\\
pUV & 0.1 & 0.2 & 0.1 & 0.3 & 0.2 & 0.1\\
    \hline
    \end{tabular}
\end{table}

\end{column}
\begin{column}{7cm}
Cálculo da covariância:

\(E(U) = 4\)

\(E(V) = 6\)

\(E(UV) = 26\)

\begin{align*}
Cov(U,V) &= E(UV) - E(U)E(V) \\
  &= 26 - 24 \\
  &= 2
\end{align*}
\end{column}
\end{columns}
\end{frame}

\setbeamercovered{transparent}
\begin{frame}
\frametitle{Correlação de duas VAs}

O \textbf{coeficiente de correlação} entre as VAs discretas \(X\) e
\(Y\) é calculado por:

\[
\rho_{XY} = \frac{Cov(X,Y)}{\sigma_X \sigma_Y}
\]

\begin{itemize}
\item
  A divisão pelo produto dos desvios padrão serve para padronizar a
  medida
\item
  Permite comparação entre quaisquer outras variáveis, pois
  \(-1 \leq \rho_{XY} \leq 1\)
\item
  Valores mais próximos de \(\pm 1\) indicam correlação forte
\end{itemize}
\end{frame}

\setbeamercovered{transparent}
\begin{frame}
\frametitle{Exemplo 5.15}

\begin{itemize}
\item
  Para os dados do exemplo \(5.5\) calcule a covariância e a correlação.
\end{itemize}

\begin{table}[h]
\centering
    \small
    \begin{tabular}{ccccccccccc}
\hline
Sub-região & 1 & 2 & 3 & 4 & 5 & 6 & 7 & 8 & 9 & 10\\
X & 0 & 0 & 0 & 0 & 1 & 2 & 1 & 2 & 2 & 0\\
Y & 1 & 2 & 1 & 0 & 1 & 0 & 0 & 1 & 2 & 2\\
    \hline
    \end{tabular}
\end{table}
\end{frame}

\setbeamercovered{transparent}
\begin{frame}
\frametitle{Exemplo 5.15}

Anteriormente já obtivemos a conjunta e as marginais de \(X\) e \(Y\):

\begin{table}[h]
\centering
    \small
    \begin{tabular}{ccccc}
\hline
X/Y & 0 & 1 & 2 & P(X=x)\\
\hline
0 & 0.1 & 0.2 & 0.2 & 0.5\\
1 & 0.1 & 0.1 & 0.0 & 0.2\\
2 & 0.1 & 0.1 & 0.1 & 0.3\\
P(Y=y) & 0.3 & 0.4 & 0.3 & 1.0\\
    \hline
    \end{tabular}
\end{table}

A marginal de \(XY\) é

\begin{table}[h]
\centering
    \small
    \begin{tabular}{ccccc}
\hline
XY & 0.0 & 1.0 & 2.0 & 4.0\\
pXY & 0.7 & 0.1 & 0.1 & 0.1\\
    \hline
    \end{tabular}
\end{table}
\end{frame}

\setbeamercovered{transparent}
\begin{frame}
\frametitle{Exemplo 5.15}

Com isso: \[E(X) = 0.8 \quad E(Y) = 1 \quad E(XY) = 0.7\]
\[Var(X) = \sigma^2_X = 0.76 \quad Var(Y) = \sigma^2_Y = 0.6\]

Assim, a covariância será:

\[Cov(X,Y) = E(XY) - E(X)E(Y) = 0.7 - 0.8 \times 1 = -0.1\]

E a correlação será:

\[\rho_{XY} = \frac{Cov(X,Y)}{\sigma_X \sigma_Y} =
 \frac{-0.1}{\sqrt{0.76}\sqrt{0.6}} = -0.15\]

\end{frame}

\setbeamercovered{transparent}
\begin{frame}
\frametitle{Exercícios}

\begin{itemize}
\item
  Seção 5.1 - 1, 2, 3, 4 e 5.
\item
  Seção 5.2 - 1, 2, 3, 4, 5 e 6.
\item
  Seção 5.3 - 1, 15, 17, e 29.
\end{itemize}
\end{frame}



\end{document}